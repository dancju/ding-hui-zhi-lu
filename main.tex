\documentclass{book}

\usepackage[fontset=none,heading=true]{ctex}
\usepackage{hyperref}
\usepackage{parskip}

\setCJKmainfont{Noto Serif CJK TC}[ItalicFont=Kaiti TC]
\setCJKsansfont{Noto Sans CJK TC}
\setCJKmonofont{Noto Sans CJK TC}

\ctexset{chapter/numbering=false}
\counterwithout{section}{chapter}

\begin{document}

\pagenumbering{roman}

\thispagestyle{empty}
\begin{center}
    \vfill
    {\huge\bfseries 定慧之路} \\
    \vfill
    {\bfseries 致光法师}\quad 讲述 \\
    {\bfseries 明至居士}\quad 笔录
\end{center}
\pagebreak

\thispagestyle{empty}
\begin{center}\itshape
    这条定慧之路,你若曾经走过,何必猜想我是谁!\\
    我本无我,常漂流生死海,何必问我在哪里?\\
    不是贪嗔痴,更不是戒定慧,我是谁呢?\\
    就是你!要休了我,请你永远记得,\\
    你越无我,我在你心中越伟大。\\
    你已经觉悟到无我了吗?\\
    无我怎会觉悟无我?\\
    是谁悟无我?\\
    悟即我!
\end{center}

\tableofcontents

\clearpage
\pagenumbering{arabic}

\chapter{第一讲}

\section{定慧基础知识}

\subsection{舍摩他与毗婆舍那}

佛教依用心的情况,把禅修方法分为两种,即所谓的舍摩他(samatha)和毗婆舍那(vipassanā)。舍摩他是修止,毗婆舍那是修观。

止和观是论里面用的名词,在经里面用的是定(舍摩他)和慧(毗婆舍那)。止是修定,观是修慧。因为众生根器的不同,有些人必须先修定,后修慧;有些人则可以直接修慧,然后产生定;有些人定和慧一起修。止和观都可以产生心一境性,心一境性也称三昧,即三摩地。三昧就是心和境合一。

在修舍摩他时,佛教讲四禅八定。佛教强调修禅定时要能够进入深浅不同的定,即入初禅、二禅、三禅和四禅。

其实,四禅八定不是佛教的专利,外道也能够修四禅八定。释迦牟尼佛未成佛之前,曾经跟外道修定,一直修到第八定——非想非非想处定。在这里,我只教大家修到四禅。为什么呢?因为在四禅八定里,后面的四个定是属于无色界的定。由于无色界的定力太深,只能修定,很难修观;前面的四个定是初禅到四禅,属于色界的定,色界的定能够修定,也能修观。

在四个禅定里面,初禅最浅,四禅最深。

初禅:称为离生喜乐地,离五盖(昏沉、掉悔、贪欲、嗔恚、疑)。

二禅:称为定生喜乐地,离觉观(觉指粗的觉察,观指细的觉察)。

三禅:称为离喜妙乐地,离喜。

四禅:称为舍念清净地,离乐(出入息断)。

舍念清净地的意思是,如果你的心念能够达到四禅,当时,你就会生起几种修行人最珍贵的心念:

1.舍的心念——就是平等心,没有造作的心,在四禅里面会升起来。

2.舍受——在四禅里面没有乐受,只有不苦不乐的舍受。在初禅、二禅和三禅里面,都有乐受,四禅的心比三禅的少了乐受贪着。

3.念清净——在四禅里面,正念最强的时候,心中完全没有杂念。所以,四禅也称为念清净。

如果一个人修到四禅,生起舍念清净的心,修观就很容易成就。有四禅的清净心,所观察的佛法就会很容易现前,即不必经过思考,当下就能被你观察到。因此,在我们打坐修行时,就应该尽量掌握和认识四禅,并且要修到四禅。假如一个人正在打坐,如何衡量他的心有多稳定(即他的定力)呢?佛教就是以初禅、二禅、三禅和四禅来判定。初禅的心念比较粗。如果静坐达到四禅,心念就会很清净。

心念的清净程度会影响修行用功的效率。如果你能明了初禅到四禅的心念差别,你就明白以微细的心去观察佛法的重要性。假如今天你的定力只能到初禅,就直接去修观、参话头,不论你多拼命修观、或参话头,初禅的心力只是会在修观时打妄想而已。如果你今天定力达到四禅,无论你参话头或者修观,你都能用很微细的心去修。四禅的功效和初禅是绝对不一样的。因此,认识初禅到四禅对修观是非常重要的。

\subsection{修止和修观的差别}

所谓的修止,是你的心念专注在某一个境界里不要动。当你专注于这个境界,任何的外境来,你都不理它,依然专注在同一个境界里面,这就是修止。

什么是修观呢?就是一个境界来的时候,你的心念随着所在境界以所修的观法去观察,但心中要明明了了,不被它迷惑,而且要看清它的真相,这就是修观。

所以,当你在修行任何法门时,如果碰到各种境界来,你都不理它,一直保持静静不动,这种修法就属于修定。如果任何境界来时,你都很专心地用佛法去观察它究竟是怎么回事,从中去觉悟,这种修法是属于修慧。因此,当你们在修观的时候,我会向你们强调:不允许入定!因为多数人入定后心就没法去观。只有定力和智慧都很强的人,他可以定慧同时修,但是这种人很少。一般人要先修定,后修慧。即先修止,后修观。

重复一下,当任何境界来的时候,你一直保持不动,这种修法就是修止。如果任何境界来的时候,你的心念随着它去,并且看清它的真相,但不被它迷惑,这就是修观。因此,修止的人,要避开动乱的境界;修观的人,他不怕任何境界;这就是修止和修观的差别。不管静坐或经行,都可以修止,也可以修观,看你怎么用心了。每次静坐时,都要明白自己是在修止或是修观。同样地,每次经行时,一定要分清楚你是在修止还是修观。在这里,我会指导你们从初禅修起,修到四禅之后,才教导你们继续修观。

\section{修定的基本方法}

修止方法很多,今天,先介绍``出入息观''。佛讲的出入息观并不是数息或随息。观呼吸的方法,在佛教里面称观出入息。发展到后来就叫``数息观'',更进一步发展成``六妙法门''。很多人把出入息观错误地以为是数息观,数息观是粗的方法,比较容易修。佛经里讲到的修法是观呼吸,不是数呼吸,只是妄想杂念很多的人需要数息。这里,我以观出入息的方法来指导大家。

根据经典的记载,观出入息的方法,就是观鼻端前面的呼吸。所以佛教里有这样一句话``眼观鼻,鼻观心'',有人把它叫做``观鼻尖白''。就是从鼻尖前来观呼吸,进一步从中观心,鼻尖白是因为观到后来在鼻尖见到光明。

\subsection{调身方法}

有些人打坐修定不久,会出现一些现象及障碍。比如姿势障碍、呼吸障碍、身体疼痛等,这些大都是因为身体没有调好所造成的。

静坐的姿势是很重要的,修禅定最好的姿势是双盘。但是入定不一定要盘腿,重要的是全身放松。优波离尊者第一次入定时是站立的,当时他在为佛剃头发呢。但是,一般上静坐要姿势正确。坐时首先不必设法摆正头,只要眼睛向前看,头就自然正。然后眼皮垂下来,
眼皮垂下时别忘了眼睛也下视,眼睛不可向前看,下视后不要理会眼睛,太理会眼睛会产生幻境。

每次上坐时,一定要检查一下全身是否放松。我说过,不一定要双盘或单盘,重点是全身肌肉放松。因为如果你静坐一小时,身上某处肌肉拉紧一小时,一小时后拉紧的部位就要疲劳。有些人坐久了,他的头就会低下来一点点。低一点点不要紧,不可以头低到打瞌睡的样子,除非你进入一种定的时候,头就会自然低下来,那是另一回事。

\subsection{调息方法}

\subsubsection{呼吸方法}

呼吸方法有胸部呼吸和腹部呼吸,胸部呼吸会胸闷气短,所以必须用腹部呼吸。若要坐久或入定久,一定要用腹部呼吸。吸气时,肚皮要自然涨,呼时自然凹进去,不可用意念控制呼吸。最好的腹部呼吸是吸气吸到换气之前,小腹会有一种吃饱饭似的感觉。为了做到这一点,一定要放松裤带,如果你穿的裤子是松紧带的,必须把裤带拉到肚脐下四指宽。胸部呼吸无法达到很微细,从二禅开始会造成气喘、胸口闷、痛,乃致无法修到三禅。

\subsubsection{观呼吸粗细}

即修行用功之处,出入息的业处就是呼吸的动作。从鼻孔到你的丹田、肚脐都有呼吸的动作,观不同部位的呼吸,对于心念的影响也不一样。道家观下丹田,就是观肚皮上下的动作,是个很粗的呼吸动作。佛家用的方法是观鼻端前的呼吸,是微细的动作。观丹田的境界不但很粗,而且心集中丹田会产生内气的运转,会带来很强的气。佛教说修定所专注的境界有大有小,观肚皮的境界是比较大,比较粗的。虽然道家修任督脉也能入定,但境界粗,要修很久才能定下来。佛家观鼻前的呼吸,它的优点在于鼻端前的呼吸是很微细的,很快就能入定,如果你的心念不够微细的话,你就观察不到鼻端前的细境界。

\subsubsection{业处:感觉人中处出入息的相}

业处就是修行的用心处,出入息观的业处是六尘中的触尘,即观鼻腔对外面呼吸的触觉,不可在鼻孔内感觉呼吸,而是在鼻腔外来感觉,就是人中这个位置。当你呼吸的时候,有风在人中这个位置吹过。因此,观出入息时不要弄错,不要观鼻腔里面,而是在鼻腔外面。因为鼻腔外面的呼吸比较细,心念粗的人没有办法感觉到。如果你叫一个人观鼻端前的呼吸,很容易发现他平时的心念粗或者细;当他轻易地在人中观察到有风吹过,就可以知道这个人很容易进入微细的心。相反地,心粗的人找来找去都没有感觉。有些人为了找鼻前的呼吸感觉,找了三天都找不到;因为他的心念太散乱,太粗了所以找不到。一下子就观察到的人,他修禅定就会很快入定。

你们现在要注意鼻端前的呼吸,一定要设法知道,呼吸给你的感觉是什么呢?就是八种触觉。

\subsubsection{呼吸的相:出入息的长短粗细冷热滑涩}

就是要观察呼吸的长短的触觉;粗细的触觉;冷热的触觉;还有滑涩的触觉。所谓观出入息,就是观呼吸的这八种变化。

长短以什么来比较呢?是以前后的呼吸来比较的。不是你和我之间的比较。即呼吸的时候,看下一次呼吸比上一次呼吸长了还是短了,是呼气长还是吸气比较长呢?如果呼吸调得很好,就会越来越长。

冷热则是观察呼气热还是吸气比较热。但是,当呼吸微细时则冷热不明显,此时就不必观察冷热了。

滑和涩是什么呢?指的是你呼吸的易难。有时候吸气比较难,好象有阻塞,这就叫涩。如果容易吸气叫滑。呼出来也是一样,也就是说我们呼气的时候,是容易还是难,这就叫滑跟涩。

粗细是你呼吸的风的强度,是粗还是弱。如果你呼吸的风声很大,那就是粗。如果风声很弱就是细。一个人如果要入定,他呼吸的风要求细。请记住:要入定不是要求呼吸长,是要求呼吸细。以下是观出入息的要点:

⒈开始时要注意的是呼吸的冷热感,心不数息也不随息。

⒉呼吸越细,冷热不明显。

⒊当呼吸微细时,就要专注在呼吸的粗细。

⒋呼吸越细,心念就越细;心念越细,入定就越深。

⒌观呼吸粗细时,不要留在同样的粗细,要越观呼吸越细。

⒍呼吸越长,就能入定越久。

\subsubsection{错误的业处:人中、气感}

你观呼吸时,不可观想将心抓住,安放在人中这个位置,这种行为叫做作意太甚。观呼吸时必须靠触觉,心专注呼吸在人中摩擦的感觉。千万不要没有摩擦的感觉,硬把心集中在人中,那就错了。换句话说,不允许将心念集中在身体的任何一个部位,包括人中,只允许集中在呼吸上。

观呼吸除了八种触觉之外,还会产生气感,即身体中有气在运转。练气功的人都知道,意到气到。中医也认为心念集中到哪里,气就会跑到哪里。任何长期静坐的修行人,都明白有此事。可是,那些不懂静坐的人,一听到有人讲到气,就说那是外道。由于我们在鼻端前观察呼吸,就会有一些气集中到人中这里来,这时心要分辨清楚,不可以去观察气感,只允许观察呼吸的风。如果你去观察气感,就发现人中越来越重,越来越麻。观察更久一点,你会觉得整个嘴唇都会麻起来。这时,你的心念除了观察呼吸,你也在观察涨的感觉。八触是呼吸给你的感觉,不是皮肤麻的感觉。在观呼吸时,人中麻的感觉不是你想要观的境界,你要清楚所观的境是出入息,身体上的痛和麻也不是观出入息要观的境界。

那些不懂得如何观出入息的人,心念往往会集中在两件事上:一件是看呼吸,一件是集中于气的麻、涨的感觉。有些人还集中第三件事,就是他多加一个心念——将注意力集中在人中的意念。要清楚:观察呼吸时只有集中呼吸的触觉才是对的,其余的集中都错。要弄明白这三件事,你才不会错用心,才能以观察呼吸的修法进入心一境性。

\subsection{调心方法}

静坐除了调身——姿势、调息——呼吸,还有调心,保持正念正知。就是正念正知专心于心一境性。

\subsubsection{心一境性}

佛教的任何修行法门,都要清楚修法所观的境界,及如何用心于境界。出入息观的修法是以呼吸为境界——触觉。心在任何时候都要保持在呼吸上,你的心所对的境界是呼吸。你心中的任何感觉必须是和呼吸相关的,比如长短粗细的感觉。你可以起心动念,不过生起的心念必须是关于呼吸的事,比如想观冷热的差别或专心数出息等等。如果你做到了这一点就叫做心一境性。如果你的心念跑去听。在当时,你就离开了呼吸——所观的境,就不叫心一境性。心一境性维持的时间有长有短。即使时间很短很短,也是心一境性。如果你在五秒钟内很集中,五秒钟内也叫心一境性。心一境性慢慢延长,就产生定,佛法称为三摩地。即你心中的每一个念头都集中在同一境界,这就叫入定——三摩钵地。

\subsubsection{经行可以克服昏沉和掉举}

有些人一静坐就会有昏沉、散乱、掉举等烦恼。凡是静坐时很昏沉不要硬硬支撑。就是说,你坐了十五分钟以后,觉得很昏沉,可以去洗一个脸。如果你很疲倦,我会让你去睡觉。为什么呢?睡眠不够的话,静坐时疲倦硬撑着,是浪费时间。不如去睡一、两个小时,然后再来坐,那效果会更好。用功静坐要讲效率的。不是说,我在这里坐一小时,我就是用功一小时,不相等的。我如果在这里坐一小时,心里老打妄想,我是在这里打妄想一小时。那么,静坐时坐得很昏沉,很掉举,怎么办?可以口含一粒糖或起来经行,经行能克服昏沉和掉举。如果你静坐一小时,前面十五分钟都很清醒,过后都是昏沉的,不如你静坐十五分钟,经行十五分钟。再重复静坐十五分钟、经行十五分钟,那效果会更好。经行也能修定,所以,我不会强迫你们一定要一起坐在这里。但是初学者在早上一定要至少坐两个小时。其余时间,你可以坐了走,走了再来坐。

\subsubsection{念念要分明}

修定的重点是不管你坐或者走的时候,都要保持正念,就是念念清楚自己的心念。当你在观出入息的时候,要清楚心始终维持在观呼吸上;除了这件事,其它的事情都是错的。也就是说,如果你在观察呼吸的过程中,忽然想去念几句咒语、或者想念佛号,都是错的。那些都是掉举的烦恼。如果你在观呼吸的过程中,忽然间看到佛,也是错的,要斩掉它。为什么呢?因为你目前的目的是要完成观出入息的心一境性——以呼吸为境。凡是离开呼吸的境都是错。

\subsection{修定时如何经行}

经行是怎么回事呢?是来回地行走着修行。行走时也可以修定,也可以修慧。就是说可以修止,也可以修观。

\subsubsection{经行的方法}

修完四禅之前,我教你们经行是以修定为主。经行的方法就是来回地走,限定在十五步到二十步之间,不可以兜圈子,只能来回走直线。在这来回走的过程中,如何保持正念呢?首先,要在身体的动作上保持正念,身体以外的动作,你去注意的话,就是错的。就是说,你的脚在走的时候,身体的动作心里要明明了了。

如果你走慢一点的话,你可以观察脚在``上\ldots 下\ldots 接触,上\ldots 下\ldots 接触''。如果你走快一点的话,你可以观察两脚``左\ldots 右\ldots 左\ldots 右\ldots''地向前走。要这样观察,左脚动的时候,心中要知道是左,右脚动的时候,心中要知道是右。此时除了左右脚的动作这件事以外,观察其它的事情都是错的,全部心念都要专注在脚的动作上。当你走到尽头的时候,你要停下来,你要知道"停"。不但要清楚知道动作,心还要专注地念,做什么动作就念什么,如坐时念"坐\ldots 坐\ldots 坐\ldots",走时念"左\ldots 右\ldots 左\ldots 右\ldots",站时念"站\ldots 站\ldots 站",转时念"转\ldots 转\ldots 转",你要一面清楚地知道脚动作,一面在心中念着动作。

为什么要念?是为了加强你的心念专注在脚的动作上,不要做第二件事情。如果你突然间注意有灯,那么就是说你的心念已经离开了脚。这时候,你快一点警告自己,要念`知道\ldots 知道\ldots'。这`知道'就是提醒自己:我的心现在跑到眼睛这里来了,快点回来观察我的脚。这样就是修定。虽然你在动中,心专注在身体的动作上,这是在动中产生定力的方法。

\subsubsection{经行时的心念要从粗变细}

在走的过程中,心念有粗有细,我们要设法达到细心。当心很粗的时候,你什么感觉都没有。当心念比较细的时候,你就能感觉到脚的肌肉在摩擦。甚至那肌肉摩擦的声音你都能感觉到,乃至骨骼里面的响声。如果你走到这样微细的时候,你的骨骼和肌肉的摩擦都清清楚楚,这时,你必定心无杂念,并且身心也会觉得特别轻安。当你心细到很集中的时候,会出现一种要跌倒的现象!你连走路的平衡都忘记了,就是说,你非常专注在你的动作上。经行来回走的当时,心只维持一件事情,什么也不管,只要知道脚的动作。当你想起另一件事情,就是妄想、杂念,你要快点念:知道!知道!然后快点摄心回来,小心看你的脚。

\subsubsection{任何时候要保持正念正知}

这样的经行,要走多久?不限时间,越久越好。能走一小时两小时更好。如果你能够这样很平静地走两小时,效果跟静坐两小时是一样的。如果你不懂如何经行的话,可能是在打妄想、散步,不是修行。如果经行走累了,你就回来静坐,坐累了就再去经行。明白了吗?这样交换地修,你的心就不会觉得坐久了,没事干而起烦恼就想休息。身体坐累了就起来走,走累了就打坐,这样的话,就可以整天地在修行。

如果整天持续地修持,不只在经行和静坐时,心只专注所修的法门。在其它时间里,也必须保持正念正知。就是任何时候你都要对自己的心念清清楚楚,知道自己在做什么。如果敲钟了,心想去吃饭,经行修得严密的人,就不会匆忙地就走去了。他会小心地看心念如何生起,从想去吃饭开始,观察自己的所有动作;就跟经行一样观察起身、站立、走去、端饭等等,吃饭时心中明明了了,自己正在拿饭碗、正在嚼食物、正在吞咽等等。心里只观照当下身心的变化,不会想第二件事情。听懂我说什么了吗?就是只管你眼前当下的动作,心中清清楚楚。不去想其它的事情。这样的话,你修定就会很快。

关于修定,我就大略介绍到这里。

\chapter{第二讲}

\section{入定的技巧}

\subsection{入禅定的条件}

\subsubsection{要离五盖}

修禅定的方法有很多种。按照我们佛教的禅定修法,不管修哪一种禅定,都要进入初禅、二禅、三禅和四禅这样渐进地修上去。要入定,就要具备一些因缘,就是当时要离五盖。所谓五盖是:昏沉、掉悔、嗔、疑,还有贪欲。当这些烦恼没有现前的时候,如果修法正确,都有机会入定。如果有这些烦恼在的话,就很难入定。除了烦恼的因缘之外,我们身体上有某些障碍,也会影响我们入定。

\subsubsection{初禅的觉受}

要入定,首先要有入定前的一些觉受。就是说,如果一个人的身心都很舒适,修法又很正确,那么,他要入定之前会产生轻安的现象。即全身非常舒服,一般人的感觉是全身轻飘飘。初学禅定的人,在入定前会有轻安现象:有些人会觉得身体越来越大,甚至整个身体都在膨胀,大到他会觉得充满这个世间;有人会觉得身体越来越小,或身体浮起来。这些都是轻安的现象,这种现象过后,就会产生很快乐的感受,这都是入初禅前轻安的觉受。在禅定里,初禅、二禅、三禅都有不同的快乐感受,到了四禅就没有乐受了。

入定的人一定会觉得全身很舒服很快乐,这快乐的感觉会使他身上原有的病痛等不舒服感消失。另外,当他入定的时候,呼吸一定是非常均匀、非常舒适的,呼吸一定会变得微细。我们可以从这些现象,分辨出入定的一些情形。根据经典说,入初禅会生起觉、观、喜、乐、定五件事,称为初禅五支,初学者入定时是分不清楚这五支的,所以初学者不必先理会初禅五支,以免分心。初学者可以从心念集中了、呼吸变微细了、身体非常舒服了,以这些现象来确定自己已经入定了。另外,初入定的人往往有一种感觉,就是他从非常舒服的感觉中出来之后,会觉得忽然脚酸、麻、痛了;但是,在他出来之前却不知道痛。为什么呢?因为初禅只有乐受没有苦受,所以,你静坐到心念很平静身体非常舒服的时候,一旦休息就感觉全身疼痛,这表示休息前你入了初禅。这是很多修禅定的人都有的经验。但是有些人不但不明白为何麻、痛,还会否定自己已经入定了呢!原因是被那些讲经教的老师误导了。那些法师不明白入定是怎么回事,将入定讲得很难很难,甚至于听到你修禅定,他就会说小心着魔。其实四禅八定,不是佛教专有的,外道也会的。为什么我们佛教一讲起禅定,就怕会修出问题呢?原因是没有正确地去认识初禅到四禅是什么。

\subsubsection{入定不是等待机会}

刚才说,入定过程有呼吸的变化,心念的变化,还有身体感觉受乐的变化。许多人不懂得怎样利用这些变化来认识入定的道路。所以,都是先安坐,然后继续坐、坐、坐,时间久了,他不知不觉地进入定里面了。到底什么时候入、怎么入,他不知道。这样的修行人往往认为必需坐得很久很久,总之,坐久了就一定会入定。其实坐久久而入定,就是不懂如何入定,他不知什么时候入了定?也不知自己怎么入。

懂得怎么修定的人,要懂得如何入定与出定。不懂修定的人,即使入了定了也不知道是入了定,就那样在那里傻等。很多出家人都能进入初禅、二禅、三禅,甚至有的出家人修到了四禅。他们的静坐经验是什么呢?原来每一次他坐下来时,总想体验他上一次静坐的觉受,然后就等、等、等时间到了,他所期待的体验又出来了,就如此入定了。大多数的出家人都是这样的,期待上一次静坐的体验而入定,这就是他不懂得如何入定,就不信一瞬间也能入定。为什么他不懂呢?原来很多人忽略了入定的过程,由于不观察入定过程最重要的讯息,只好等待所体验过定中的舒服感觉,当那体验出现时他才认为:哎呀,我又在定里面了。这说明他没有观察入定的过程,所以,他每次都不知不觉地入禅定。

\subsection{入定最重要三事:心细、息细、乐受}

在这次的禅定学习里,要学习认识入定的过程。其实,过程很简单。为什么说很简单呢?因为在禅定里面,有快乐的感受,呼吸很细,心念也很微细。就凭这三件事情,我们可以这样说:当你入定时,呼吸一定是从粗变细、心念从粗变细、身体从没有乐受变成有乐受。就是说,入定的过程中,至少有三件重要的事情在变,你要小心观察它变化的过程。

\subsubsection{如何观察出入定时的三事变化}

要什么时候观察这三件事呢?当你静坐时,一旦觉得呼吸很细,没有杂念,心念很平静,全身有乐受出现时,你不要留恋在里面,要快点退出来。为什么要如此呢?因为当你发现身体快乐,心念变细,呼吸变细的时候,你就能在退出时观察它的变化。你会发现,心念变粗,呼吸变粗,快乐的感受在退。当乐受完全退时,你就快点再度集中,先想我要再度入定,然后再度集中于修法。如此,你一定有能力再回到原来的觉受。也就是说,你刚刚从初禅下来,你一定有能力再回到初禅。你再度集中的当时,就快点注意观察三件事的变化,呼吸变细,心念变细,还有乐受又出现了。这是非常重要的三件事,因为这过程就是入定的道路。所以要你去观察心念怎么从粗变细,呼吸怎么从粗变细,怎样从没有乐受变得有乐受。这就是你自己要去认识的道路,什么时候观察此三件事?时机(火候)就是当呼吸很细,没有杂念,心念很平静,全身有乐受时。

每次一静坐,你就要想:啊,我现在就要入定了!初学者一般是做不到的。你必须坐一段时间,坐到乐受快出来了、没有杂念了、心念微细了、呼吸微细了,这时候快点退出来。一退出来了你就想:啊,我现在就要入定了!这时你一定办得到。在入定过程要观察三件事在变化:心念、呼吸和乐受。

\subsubsection{重复练习出入定过程}

这样小心重复观察入定三件事情:心念、呼吸和乐受,你就会懂得原来入定过程是这么回事。必须重复训练,上去,下来。再上去,再下来,做越多次越好。以后,你就会越来越快地入定。一般人不懂得重复训练入定,只会贪着乐受而住在定中的乐受而不想出来,一直呆,呆到定力退了才出来,如此修定者能入定却不懂怎么入。所以,任何人初入禅定,千万不要一入了定就不出来。应该是一进去就快点出来,然后,再快点进去快点出来。一直重复做,做到你很熟练了,很清楚如何入定后,才来加强定力,所谓加强定力就是进去了不要马上出来。在练习加强定力时,要在定里多久呢?初学者入初禅千万不要太久,五分钟就好了。在里面呆五分钟就要出来,但是不要下座,然后再进去五分钟后出来。为什么呢?因为在初禅里呆得太久,心念可能会更细而离开初禅。甚至于深入到更高禅定,呼吸更细,心念更细,也更快乐。于是你对初禅心念的粗细混乱不清。也就是说练习加强初禅定力时,他在入初禅半小时内,要出入定六次。这样不但修了半小时的初禅,而且懂得出懂得进。当每次出入定五分钟做得很熟悉,很有把握了,你就去入定十分钟或者十五分钟。十五分钟有把握了,你就去坐一小时。就这样地加强定力。当你觉得这个定很稳固了,然后才能设法进入另外一个禅定。这就是所谓入定的技巧。

\subsection{解除入定的障碍}

刚才说修禅定有一些障碍。一个是五盖的烦恼,另外是身体的某些障碍。

\subsubsection{气脉阻塞——造成身上的疼痛}

关于身体的障碍,一个在胸前,一个在后背。在静坐的时候,身体里会有气的运转。就是说,当你精神专注在你所观的境时,你必然会全身放松。身体一放松,你身体的气就会运转起来。气功师会说是在练气功。其实不是,我们在专注修定。但是因为你的心专注在一个境里面,对身体不理会,你的身体就会放松,身上的气就会运转起来。当它运转的时候,如果你身上有一些气脉阻塞,气运转到那里就会疼痛。于是禅定就修不好。一般受寒的阻塞都在后背,一般呼吸或心理的问题会造成前胸阻塞。忧郁、劳心、嗔心等就会感觉胸口闷。还有一个就是胃的部位,有些人胃寒,他坐到一定时候就会打嗝。

当你静坐到身上气感发动时候,若身上有疼痛,而不是酸痛,表示你静坐坐的好。为何说好?原因是身上有病痛,你平时不知道,你静坐的时候,气要打通病灶而痛。你要去处理病痛,如果你不去处理,气就会干扰你。如果你不去处理它而每天坚持坐,慢慢的也会自动打通,但是要花很长的时间。如果你们发现任何的疼痛,都要将它处理掉,不要认为出现病痛是老师教错,或者是修错。过去在静坐时若受到惊吓,以后每次静坐就会胸前痛,你要找医生或者气功师帮你调理,以免继续干扰静坐。要记得,在静坐的时若受惊吓,不要立刻就睁眼动身。应该静下来,吞口水或者将气引到丹田,过后找人处理。

背后有几个部位,就是会阴、命门、肺腧、大椎、玉枕等。静坐时会有气通过这些脉轮穴位,如果这几个部位阻塞会造成一种冷热现象。一个是在肚脐背后的命门冷,会导致腿冷麻。还有胸口背后的肺腧穴一旦受寒,就会冷疼。感冒的时候,大椎往往会冷痛阻塞。如果是大椎阻塞,你静坐的时候,会觉得背后很热很热,但颈项以上凉凉的,这是气不能上来的缘故。如果是脑后的玉枕阻塞,你会觉得整个颈项发烧,头重重的。这也是气不通、阻塞。有这种现象,你一定要去处理它。处理不了,只要长久地坐,虽然被困扰一个时期,最终它也会通。

当气自动调理过去造成的病灶而痛,说明你静坐有进步,所以说坐久出现痛是好事,不明白的人就会因痛打退堂鼓。

\subsubsection{头上留气——久了造成头痛}

大多数人静坐几天后太阳穴和眉心这一带痛胀,这是静坐时不小心造成的。为什么呢?因为静坐时气会升到头上,很多人没有觉察有气留在头上就休息了,一次留一点点,用功几天后,就会疼痛。情况严重时,嘴唇裂,舌头生疮,睡不着,虚火上升。这些都是气留在头上造成虚火病气。好多出家人都遇到这方面的问题,自己被气干扰成病,对修炼时的气无知而修成``虚火外道'',却骂气功是外道。

所以,每次静坐之后,头上某些部位会有气,你如果有把握处理这些气的话,可以用任何方法处理,处理气的阻塞是不分佛道或外道的,外道有好办法也可以采用。不然的话,最好循古人的规矩,静坐完了之后,先搓热手掌,以爪梳头、以掌洗脸、以指按摩身体手脚。按摩的时候要注意,不要立刻睁开眼睛。按摩完了之后,才能睁开眼睛。

按摩就是让你将积在头上或身上的气疏散掉。另外有些人,静坐完了就去睡觉,这是要不得的。这样容易造成头疼。如果这些问题你都懂得处理,你就不会被气干扰。不然,你静坐到最后,你会因周身不舒服而心灰意冷。

静坐到了一定时候,敏感的人都会发现有气在运转,大多数人都经验过太阳穴会发涨。这是为什么呢?原来我们修心养性会产生清净的气,贪嗔淫欲会产生污浊的气,清净的气向头上升,越清净的气升得越高。污浊的气往下降,越污浊的气降的越低,最高到头顶百会穴,最低到小腹下的会阴穴。所以当心念清净到接近初禅的时候,你身上清净的气就会升到眉毛的这个水平。是一个水平,不是眉心一点。在二禅的时候,会有一股清净的气升到发际这一带,就是头发和额头之间。在三禅的时候,会有一股清净的气在百会里。如果你到四禅,那股气就会在头顶外面了。细心的修禅定者都会发觉这种现象,就是静坐后这几个部位会有气。因为不了解这种现象,所以不晓得处理这些气,造成有些人会有不同部位的头疼。

\subsubsection{回归平时状态——下座前要按摩头和身体}

你每次静坐完了,一定要让这些气降下来。有些人很敏感,他会知道,有些人不敏感就不知道。不敏感的人静坐之后,要好好的按摩,以避免气留在头上。如果你用按摩处理不了,就要做一些观想,观想气慢慢地从头上降下来。你可以用手掌心面对自己的头慢慢地慢慢地向下拉。向胸前中间拉,拉到丹田,重复做这个动作。如果还是不行的话,你可以拍打来处理留在头上的气,用空心掌拍打。拍后会觉得你头上的气粘在手上,要将它甩掉。如果这些你都懂得处理,要进禅定是不难的。有些人已经坐很久了,虽然没有什么杂念了,总是不能入定,没有乐受。为什么呢?多数原因是身体有病,造成心无力集中,于是没办法入定。因为禅定是很强的心力集中,如果心力集中达不到相当强度,就没办法入定。也就是说你的心力无法集中到所需要的能量,所以,身体比较虚弱的人,静坐前就要吃一些补气的药。

\subsection{静坐的气场}

很多人有这样的经验,他到某个地方很容易就心静下来了。总觉得到这个地方静坐很快就入定了。其实是那个地方的气场很好,对他有帮助。虽然外在的气场对修行有帮助,但是,靠外在的气场,不如靠自己本身的气场,就是长期静坐后,你也会形成本身的气场。至于气虚的人,要吃一些补气的药来加强,他的静坐才会进步;身体健康的人就少吃补为妙。

你静坐的场所不要整天换来换去,如果你在家里,最好每天在同一个座位坐,你会在此座位上形成一个气场。以后你再回到同一个座位,就很快能够定下来,这是环境的影响。还有,你周围的同学坐得好的,靠近他你会沾光。就是说,他的气场对你有帮助,而你的气场对他会有干扰。如果他的气场很强,你对他的干扰就会很微小,没什么影响。如果他本身的气场不是很强的话,你的气干扰了他,敏感的人就会心烦了。有病者的气场,会干扰身边的同修者,所以,当你觉得坐在这个位置非常不舒服,可能换个位子就好了。

\subsection{修定时间的长短}

修禅定一座要坐多久?不一定非要坚持坐完一枝香,为什么呢?如果你没有昏沉、掉举、散乱,你就必须坚持。如果你有很严重的昏沉、掉举、散乱,那么,你就要自己去衡量时间。如果开始十五分钟坐得很好,过后昏沉、掉举、散乱,每次都如此,我劝你不要坚持。应该怎么做呢?就是坐走交叉修。因为静坐开始的五分钟你坐得很好,十五分钟以后的效率就差了。那么,你就坐十分钟,经行十五分钟,然后再坐十五分钟,再经行十五分钟。如果你这样修,也就是说四个十五分钟里,你都能正念清楚的,那样修行就很有效果。如果说,你坐了十五分钟之后,因为精神不好,心无法集中地坚持到一小时,那是浪费时间。不要听人家说,硬要坐完一柱香就是好事,不见得啊!每个人都不一样,如果你今天精神很不好,睡眠不足,或其它原因造成疲劳,那么,我劝你快去睡觉。如果不是烦恼而是精神疲劳,睡饱了再来坐效果会更好。

入定是否入得越久越好呢?不一定!如果你要深入禅定,就必须坐得越久越好。如果不是,千万不要坐得太久。为什么?禅定坐久了,会贪。贪什么呢?就是他每次一进去就不想出来。一旦你入定就不想出来,这就是贪。因此,修禅定时,在入定之前最好先规定出定的时间。假如你要进初禅,你就自我规定:现在我要入一小时的初禅。这样,你入定后坚持一小时之内,不要上更高的定,也不要下。如果我现在要在入初禅十五分钟,那么十五分钟后一定要出来。

\subsection{修禅定有三种自在}

就是入定自在、出定自在和在定自在。

什么叫入定自在?入定自在就是任何时候我想入哪个定,就能够入哪个定,这叫入定自在。比如我要进初禅,就一口气之间进初禅。我要进三禅就一口气之间进三禅,这叫``入定自在''。如何是``在定不自在''呢?如果我要留在初禅,心念老是要溜上二、三禅,结果自动到了三禅,这是你的初禅在定不自在。什么叫在定自在呢?如果我要入二十分钟,我进去出来,就是二十分钟,这叫在定自在。如果我说进去二十分钟,一小时了才出来,就是贪着禅定。明白了吗?就是说,你入定前,要定下我现在要入定多久,然后到那个时间出来就不是贪。如果你要二十分钟,结果是一小时才出定,就是贪。明白吗?也就是说你不自在。出定自在就是坐禅的人想出定,一想出来就出来了。身心就恢复到入定之前的状况。不要以为一睁开眼睛出定,就能身心恢复常态。如果你出定后会觉得头发涨,被气锁住,很不舒服,这是出定还不够自在。所以,修定有所谓的出入定自在和在定自在。当你们熟悉了各种禅定之后,你每次入定之前,最好给自己预设一个时间。

\subsection{禅定差别}

\subsubsection{近行定与安止定的差别}

刚才说道,禅定有一、二、三、四禅,我用登楼比喻,一、二、三、四禅就是心集中的能量高低。犹如你上一幢楼,上一楼、二楼、三楼、四楼,表示说一禅、二禅、三禅、四禅,那是不同的高低的集中力,也就是心的能量。当你的心念达到入禅定之前的集中力,称为未到地,或叫做近行定,有初禅的近行定、二禅近行定。近行定是什么意思呢?近行定好象你上楼梯到某一层,因为还未进房间,你可以继续往上爬,也可以爬下来。入根本定也叫安止定,进入根本定好像进了房间,你就不能上下爬了。在近行定能上也能下,在安止定不能上也不能下。初禅近行定是心念的集中力达到了初禅的水平,不等于你当时进入初禅,所以上到初禅近行定,不等于进入初禅安止定。当你进入初禅,会有进入的感觉,整个人沉入在里面。这时若要上下,就必须从初禅出来,你就会有从里面出来的感觉。就是说进去和出来犹如进出房间,上去和下来犹如上下楼梯,是不一样的。

\subsubsection{禅定之间的觉受差别处}

初禅与二禅最大的差别是:初禅的心态有觉有观,很容易被声音干扰。初禅的人听到声音,心就乱了,所以佛说声音是初禅的刺。二禅的心态无觉无观,声音的影响,你都如如不动。

三禅和二禅的差别是:在三禅,你会觉得身体不存在了。但是,感觉头还在,觉得全身很快乐,却不知道身体在哪里。到了三禅呼吸很微弱,有些人会觉得呼吸困难,主要是呼吸不正常及胸口有毛病而造成的障碍,一般是忧郁、易怒及紧张的烦恼造成气结檀中穴。有些人到了三禅心脏会跳快。为什么呢?因为,三禅的呼吸很微细。那些心没有力的人会觉得心脏负担不了,会跳得快一点。这样,他就应该在三禅多呆,慢慢地适应,最好是吃补心气的药。

上四禅,呼吸就要停止了。有些人会呼吸停止不了。他就只能留在四禅近行定,无法进四禅。你如果深入四禅,外面的声音是都听不到的,那是最好的四禅。如果一个人在四禅里面,听不到声音以后,这个人就可以进一步修第五个定。如果你只修到四禅近行定,还会听到声音,就没有能力进入第五个定。

禅定越高,定力越强,心念越细,呼吸越细,感受越快乐。但是,从三禅进入四禅有一个很明显的现象,那就是一旦你从三禅进入四禅,快乐就会立即消失,完全没有乐受——舍受。如果你们有经验过四禅,就会知道叫舍念,就是没有造作的平等心,四禅的心不造作,而且非常清净——念清净。

\subsection{制心一处无事不办}

如果从初禅到四禅,你都弄清楚了,以后不管你修密宗、禅宗,任何修法,你都可以用禅定来判断自己的心。修禅定好象是磨刀一样,切东西的效率,要看那刀磨得有多锋利,修行时心的效率就是入定有多快。如果今天你坐下来,连初禅都上不了,然而却要修大威德金刚、或者修大圆满、或者修禅宗,这样修任何法门都修不好的。你应该知道自己是以散乱的心来修。明白吗?但是,如果今天你能进到四禅,那今天修任何法门都很有效,因为你清楚今天的心力,是以清净的心来修行。也就是说,不管你修佛教任何法门,用不同的粗细的心力来修行,得到的功效是不一样的。四禅的清净心,是修道人要去争取到的。所以,如果你的心无法平静就去参禅,你只是在胡思乱想,如此参话头打禅七,是胡打,变成烦恼纠缠不清的``缠七''。为什么呢?心都不能安定下来,烦恼一大堆,是烦恼在参缠,还以为是参禅。修任何法门最终是要修慧要觉悟,而修慧之前要有定力。佛法说有慧没有定叫狂慧,狂慧的人烦恼很多,还说自己比他人有智慧。因此,把禅定掌握好的人,再去修炼任何法门,都是有所帮助的。如果一个人他不认识到定是慧的基础,定力不足就去修行高深法门,那么,他根本不懂自己用什么心去修。我在这里再次强调:一定要修好禅定。

\subsection{问答}

问:您刚才说,身体有障碍的话,要把它消除掉才能继续坐下去?

答:如果静坐时,身体上有一些障碍,应该驱除它再来坐。

问:不能坚持坐下去吗?

答:你可以坚持忍疼,坚持下去只是自讨苦吃。比方说,脚很疼,你可以坚持不管它,硬忍,忍忍忍,忍到下座。但是不如不要忍,你快点按摩将它放松了,然后就容易再入定。

问:这样入定之后就不知道疼了吗?

答:对。入定就不知道疼痛,出定后又疼了。出现疼痛说明你已经不在定里了,因为在定里不是乐受就是舍受。疼痛既然不能入定,何苦忍半小时!不如把疼痛处理掉,可能五分钟后就能入定。

\chapter{第三讲}

\subsection{对禅定的错误见解}

\subsubsection{不知入什么禅定}

在修行的过程中,修定是为了得到清净的正念。不同的定有不同的微细和心和集中力,越高的定,心力就越集中,而且心念越清净、越微细。初禅心念比较粗,四禅比较细,修禅定是为了要争取到微细的清净心。所以,在修禅定的过程当中,我们要练习去分别什么是初禅、二禅、三禅和四禅。在我自己还没有认识这些禅定以前,因为没有得到老师的教导,曾经盲修瞎练。后来我认为,我一天应该坐上十二小时。一座坐上四小时,然后,休息吃饭,再来坐。天天这样坐,坐来坐去,有时入了定也不知道是什么定。这就是对禅定无知地盲修瞎练。我到处问人,没有人给我讲清楚。直到有一天我碰到一位禅师,我问他:你在教导人修禅定吗?

他说:是。

我又问他:你教人家修禅,你懂不懂你的学生入了什么定?

他说:当然懂了。不懂的话,不能教啊。

我听了就非常欢喜地向他学了。因为我到过好多修行道场,也找了很多人,问了很多人,他们就只教你打坐,但是分不清什么是初禅。有的法师说:哎呀,不要修禅定了,会着魔的,还是念佛吧。当我跟那位禅师学了之后才知道,原来禅定中的初禅到四禅是如此这般。我才知道以前一天坐上十二小时是盲修瞎炼、浪费时间。为什么?原来一座坐四小时,可能你只在初禅,也可能你在二、三、四禅,看你在那期间定力多高,当时你就每次回到一样的定。只知道入回同样的定,却不知道自己定到什么程度,也不懂如何前进。这种情形犹如进入深山老林,迷失方向再也没有作为了。

根据那位禅师的说法,如果能够入定,入初禅五分钟也算初禅,入初禅四小时也算初禅,不在于入定时间长才算入了深的定。如果入四禅一分钟也算入四禅。因为我们被误导了,包括我自己曾经也是这样子,以为入定是要坐很久才算数。为了说服大家,我举一些例子。

\subsubsection{经典里的入定例子}

第一个例子就是优波离尊者的故事,他在给佛剃头的时候入了定。因为是佛的头,他非常细心地剃,剃着剃着就不知不觉地入了定。佛就说:你现在是在初禅,过一阵,佛又说现在是二禅,优波离尊者就这样慢慢入定,他当时站着拿着剃刀,眼睛集中在佛的头发,心里细心地盘算着如何剃好它,耳朵还能听到佛的开示,就这样入着定。在这个例子里,我要大家认识几点,优波离尊者当时并没有盘腿,没有闭眼,还有念想,还听声音。可是,他是一心一意地专注在佛的头发,这就是入定的关键所在,他达到了心一境性。所以,不要道听途说,以为入定时是什么心念都没有、声音也听不到。

第二个例子是目连尊者的故事,入禅定是可以非常快的,快到怎么样呢?有一次,目连尊者去降伏一条龙,他用神通将自己变得很小,钻进龙的鼻孔里。这条龙想把尊者赶出来,就呼气,当那条龙一开始呼的时候,目连尊者就刹那间入了四禅。然后,龙呼也呼不动他,因为尊者在里面入四禅,它怎么呼也呼不动。就是说,尊者在它呼的那一刹那就入定了。所以出入禅定是可以非常快的。

在我跟那位禅师学时,他说:在吸一口气间就能入四禅。多数人认为要慢慢地,好久好久才能入四禅,这些人已经被误导了。也就是说,你懂得修的话,入禅定的速度会非常快,也能非常快地出来。你可以入四禅几秒钟后立刻就出来了,这也算入四禅。但是不懂的人会说,这算什么四禅?入几秒钟就算四禅了吗?他们说禅定不可能是这样啊!其实,你当时入了什么定,就是什么定,不在于时间长短。另外,很多人会以为入了禅定,不能想东西。这是很错误的。还以为入了禅定后,外面的声音也听不到了,这也是极大的错误。其实,优波离尊者是一面听佛指导一面集中心力而入定的。

我再讲一个戒律里的故事。有一次,目连尊者入了无色界的定。根据佛法说,无色界是没有物质的,没有物质,你就不可能看到东西,不能听到声音。但是目连尊者向一些比丘说,我入了无色界的定,当时听见很远很远的恒河里,有大象的耳朵拍水的声音。那些比丘就说他打妄语。为什么呢?因为那些比丘精通佛法,知道无色界里不可能听到声音。目连尊者坚持说,我进了无色界的定时听到了大象拍恒河水的声音。然后,比丘们就拉他去见佛,说他打妄语。佛就问目连说:你真的入了无色界的定了吗?目连尊者回答是。佛又问他:你真的听到那大象耳朵拍水的声音了吗?他又说:是!结果,佛还是同意目连尊者入了无色界定。佛解释说:目连尊者是入了无色界的定,但是他能很快出来又立刻进去,出来的时候,他就能听到那个大象拍水的声音。所以,在那短短的时间里面,他出了定就入回去,但他没有弄清楚他已经出来了一下。明白吗?这个例子就是说:我们在修禅定的过程里面,会有这样的现象,你从定里面出来一下立刻又进去。就在你出来那一瞬间,可能你会感到脚疼,能够听到声音,但回到四禅又没有声音了。这是可能发生的。所以,千万不要听那些论师的说法,认为一听到声音就不曾入无色界定,然后依理论否定自己。不是!确实是听到。但是,你是暂时出来听到,然后又迅速进去。那样你还是听到了。明白吗?

\subsubsection{不可按图索骥、盲目相信权威}

由于对禅定的种种误解,造成很多人没有办法分辨到底有没有入定。为什么呢?比方说,依《解脱道论》的说法,是逐渐听不到声音,到四禅的根本定就完全听不到。有人误解《俱舍论》,说到了二禅,就听不到声音了。其实《俱舍论》是说初禅天以上的天人没有吃东西,没有鼻舌的知觉。但是一个四禅天的天人想要看东西,他不但可以看,而且还可以看整个大千世界呢!乃至四禅天人要听声音也可以听。那是怎么回事呢?在《俱舍论释》里有这样的解释:四禅天天人可以用天眼、天耳,但是必须从四禅的心降到初禅的耳识、眼识,来听、来看四禅天,换句话说,四禅天天人可以在初禅到四禅之间动心念。如果你不明白同一禅天天人有不同的定心,那不重要,总之,《俱舍论》说四禅天的天人是可以看东西、听声音的。也就是说,四禅天天人可以生各种的心,并且出进不同的定。可能你不知不觉地从初禅出来一刹那,觉得腿疼又立刻入回初禅,于是觉得腿不疼,有这样经验过吗?不要因此怀疑没有入定,要以依据经典的说法,要相信出进禅定会快得你分辨不清楚。

还有,在中国佛教有一个很糟糕的说法:初禅念住、二禅息住、三禅脉住、四禅灭尽。三禅连脉都停下来,四禅已经是灭尽定了,这是极大的错误,其实,息住是四禅,虽然有脉停这回事,但是佛经里没有说这点,灭尽定是第九个定,已经超过四禅八定了。误导好多出家人,入了定还认为没有入定。

那么,经典对禅定的说法是如何呢?

\subsection{禅定的正确见解}

初禅离五盖,离昏沉、掉举、嗔心、淫欲心、疑心。有觉有观、有喜、有乐、一心在定。

二禅离觉观,离初禅的觉观,无觉无观,有四支:内净、喜、乐、定。

三禅离喜,离二禅的喜。有四支:行舍、正念、正慧、乐、定。

四禅离乐,离三禅乐,出入息断。有四支:舍、念清净、不苦不乐、定。

根据经典的说法是这样的,就是说,从禅定里面一步一步暂时降伏一些烦恼。当一个人进入初禅的时候,在定里不会有昏沉、掉举、嗔心,若有嗔心就入不了禅定,当时也不会生起淫欲心。所以说初禅离五盖。其中的疑盖是疑些什么?疑就是对自己怀疑,对老师怀疑,对修法怀疑。那样,你也进不了初禅。一般修禅定的人不懂什么是初禅,因为没有经验过初禅的觉受,所以便无从知道。用什么方法可以认识初禅呢?可以如此观察,就是任何修禅定的人,坐到一定程度后会觉得心定下来了、没有杂念了、全身很舒服快乐。其实,这就是入了定,但是不知道入什么定,要如何知道?根据佛法说,初禅有定、有喜乐的感受,当时身体不可能会有痛苦的感受。所以,如果修定时一直觉得身体疼痛,这证明当时你没有在定里。如果你静坐时,觉得心定了好长时间,身体也不觉得疼痛,突然觉得身体疼痛。这疼痛表示你出某个定了,而出来之前是入定的。为什么呢?原来是静坐者气脉未完全打通,入定久了身体就会疲劳,于是腰酸、腿痛,但是在入定时不知道。因为在定中只有喜乐的感受,从定中出来后,才觉得腿很痛。

\section{如何修到四禅}

\subsection{初禅}

初禅是很粗的定,粗到什么样呢?佛曾经形容——声音是初禅的刺。一个人在初禅里如果听到外界的声音,心里难受到好象耳朵被针刺而出定了。初禅经不起声音的嘈杂,可见是很浅的定。初禅虽然很浅,还是禅定,这个定有声音干扰就退出来了。因此,好多人不信这也是定。未修好初禅的人,无法到嘈杂处坐禅,声音一干扰,他的心就不安了,所以修初禅的场所不可以有声音干扰。但是,修到二禅的人听到任何声音就如同没听到。因为他非常专注,知道有声音却不知道其内容。

如果你们坐着坐着,觉得全身舒服了,你可能不知道这是初禅,你就先暂时假设这是初禅嘛!为什么我这样说呢?总之,当时你有定、有喜、有乐嘛!既然如此,就是一种定了,就假设它是最低的初禅。入定之后怎么办呢?前一讲,我说过了,初入定的人千万不要在定里呆太久。凡是不懂修禅定的人,他初入了定就想要呆太久,这是绝对错误的。初入定的人,一旦进去,要快点出来。要在出来的那一小段时间注意一切变化,很多修禅定的人,没有注意如何入定出定,以后就不记得如何入原来的定。所以,初入定时要快点出来,要注意出来之时,心会变粗,呼吸变粗,快乐的感受退掉了,你一定要观察这三件事。当你再度集中心力入定,你可以动个念头,心想我现在要进入初禅,然后再专注在原来的修法——观呼吸。并注意三件事情:呼吸变细,心念变细,乐受出来了,于是又入定了。如果不懂入什么定,先当它是初禅来修,不理会别人的看法。

\subsubsection{一定要先训练如何出入定}

这样出出进进初禅,越训练越熟悉,以后,你可以心想入初禅就入了,那些修过的同学都有这些经验。总而言之,懂得修禅定的人,不要立刻想我要定得长久,我要加强定力,一定要先训练怎么进和出。对于不懂入了什么定的人,千万不要想定得长久,这是绝对错误的。当你出定入定分辨很清楚以后,才在定里留久一点以加强定力。当你加强到定力稳定了,就有信心说:我的确进入了定。而且知道只要心念如此集中,我要入那个定就能入。只要我放松集中,我就出了那个定。如果你还是分辨得不清楚,可以依靠经验过的人在你入定当时告诉你。有时我也会在你修定的当时告诉你:现在你在哪个定。

初学的人多数只入初禅,为什么呢?这里面有个方法上的奥妙,就是开始能入定时不要入太久,五分钟就出来。因为任何初入禅定的人,五分钟内绝对上不了二禅,只能到初禅。就在五分钟内出出进进,那样是没办法进入二禅的,所以不必问他入什么定,他最多只到初禅。就是说,如果我们把时间限制在五分钟,入定五分钟后一定要下来,保证他只在初禅。

但是如果那个人原本就修过一些定,他可能已经能入初禅以上的定,这就另当别论了。就要问他每次静坐的时候,坐多久才觉得定下来?如果他十分钟就能入定,那时就要告诉他练习初禅时,入定一分钟就退出来。明白吗?让他不够时间进入更高的定,于是他就入定不深,最多到初禅。当他在这一分钟内认识初禅清楚后,才叫他加长时间留在初禅,此时注意,避免心念太细而入更深的定。就是说,通过时间的限制,不会进入太深的定而超过初禅。当在初禅五分钟出入熟练了,就叫他修十五分钟的初禅。这十五分钟是加强初禅的定力,初禅心粗,心不想久留,于是心就会想快点入二禅。

\subsubsection{不可以一上座就想入初禅}

初学的人不可以一上座就想我要集中上初禅,上不了的。为什么?因为平时的心是散乱的,每次坐下来,你需要花一段时间,让心念平静下来以及调好呼吸。所以初学的人练习初禅的时候,他必须坐下调心调息,不可以一坐下来就想上初禅。如果他急着入定,他就错了,他上不了的。他必须坐下来让心平静,慢慢地进入定,然后退出禅定,这时才可以想我要上初禅,而且一定可以上初禅。因为,一个人刚刚从初禅出来,他一定有本事回去。就是因为他能再度进入,他就有信心:要进就进、要出就出。初学的人,今天坐到了初禅,以为明天一坐下来立刻就能上初禅,是进不了的!他就会怀疑所修的一切了。所以,以后如果你很久没有静坐,定力退了,不可以一坐下来就想要进初禅。进不了的。你必须坐到心念平静了,觉得入定了,退出去再上去,你才能做得到。这是因为你必须让你的心从散乱状态平静下来,因定力不足而需要一段时间。

\subsubsection{如何加强定力}

如果一个人在一个钟头里面,可以入初禅四次,每次十五分钟,那就是说这个人的初禅很稳了。因为这四次加起来是一个小时。如果他以后要坐更长时间,他可以加长时间坐一小时也可以。就是说他要入初禅一小时,两小时都可以。但是,当入初禅很久很久以后,大多数人会自动地到二禅去,从初禅自动入二禅的初学者这时会分不清初禅和二禅。所以,在开始训练的时候,我不会鼓励你们花一个小时或两小时去体验初禅。为什么?因为如果那样的话,你的心念就会慢慢地从初禅进入二禅。因为是慢慢进,看不出变化,由于入定时是初禅而出定之前已经是二禅,你却还以为是在初禅。以后你总搞不清入了初禅还是二禅。初学者入初禅太久会有这种现象。

\subsection{如何修二禅、三禅}

\subsubsection{先要有足够的火候}

如果你上了初禅,什么时候才可以上二禅呢?如果有老师指导,老师知道你的初禅定力够了,就叫你继续前进。如果没有老师指导你,怎么办?根据佛法说,如果一个人的初禅定力不够,然后又想上二禅。那么,他不但进不了二禅,连初禅也会失去,这样两头都失了。就是说你初禅定力不够,你想入二禅是入不了的。

所以,你必须初禅坐得很稳了,才能修二禅。所谓稳就是说你能在初禅里坐上一小时。但是,在入二禅之前,千万不要在初禅里超过一小时以上。为什么?因为那样你的定力会加强,可能不知不觉地到了二禅,于是你就分不清初禅二禅了。所以,你最好每十五分钟就下来,下来以后再上去,这样持续加起来一小时。

\subsubsection{心离开原有定境的觉受}

如果你初禅很稳了,想要上二禅,就要离开初禅的根本定的细心,从原有的修法进入更细的心。禅定越高,呼吸越细,心念越细、更加快乐。因此,你要上二禅时必须远离初禅的觉受,心再度专注于修法上:很集中地去看呼吸,重点在于观察呼吸比初禅定还要细,心念更细更乐,然后想再上二禅。

\subsubsection{再度专注于修法上}

为什么必须再度地专注于修法上呢?就是说我们入定之后,往往有一种现象,在定里面产生喜乐的感受、身体感觉很舒服。如果你一直坚持在那种舒服的感受里面,你就跳不出那个定。每当你要升更高的定时,就必须放弃原来的觉受。就是说,你不要执着原有禅定的境界、禅定的喜乐。一定要放弃,惟有你放弃了,再度集中你原来的修法才能上二禅。如果你不放弃那些觉受,即你不愿意放弃原来的定——初禅,你就无法上二禅,正如你不愿意离开一楼,怎么能上到二楼呢?因此,当你要上更高的定时,千万不要留恋原来的任何觉受,而要更加专注地观察你原来的修法,如果你是观呼吸,就更专注地看呼吸,慢慢地就能上到二禅,二禅的变化和初禅一样,心念变细、呼吸变细,也更加快乐。这些步骤要一步一步地进行,上三禅也是一样的。

\subsection{如何修四禅}

上四禅就有些不一样,四禅是颠倒过来的。怎么讲颠倒呢?就是说更高的禅定是越来越舒服,越来越快乐。但到了四禅,快乐统统消失。此时,呼吸就断了,舍念清净的心生起来了,这就是四禅。有些人的呼吸不正常或者执着呼吸,入四禅前他会觉得呼吸要断却断不了,那种要断气的感觉很辛苦,有时会断一会儿气,又呼吸几下。这些人修四禅就不稳定了。

前天,我说过,心有力量上到某个禅定,不等于进入那个禅定。上禅定好象上下楼,在近行定时,心念可以上上下下地变粗变细。入禅定好像进入房间,在安止定时,心念不上也不下。所以,当你修到呼吸要断不断的时候,就是上四禅的时候了。当你的四禅稳固的时候,呼吸就断了。当修四禅到达呼吸断的时候,应当加长在定的时间,那样定力会继续加深,慢慢地连声音也听不见了。一定要到声音听不见,才算深入四禅。如果你在四禅里面还能听到声音,犹如你爬上四楼,只在房门口,却没有进入房间,这样只是四禅的近行定。

\subsection{四禅近行定就可以修观}

依据论典说初禅离五盖,便可以修观了。而我要求大家至少修到四禅近行定,才允许他修观。为什么呢?因为这时他有能力进入微细的舍念清净心了。但是,此后还是要把四禅彻底修好。凡是修完四禅的人,他就会明白,不同的定力,心的粗细不同。弄清楚这些心以后,就可以修观或修第五个定了。

\subsection{无色界定与超越禅}

第五个定叫空无边处。如果一个人修到四禅后,可以接着修第五个定,他必须先入到连声音都听不见的四禅安止定。如果他还能听见声音,那么他还在四禅近行定中。这时想要修第五个定,那是在打妄想。为什么呢?因为空无边处定是属于无色界的定,你入了那个定,是绝对听不到声音的。当你把四个定的心念都分辨得很清楚了,便能够做到依心念入定,想入哪个定就能即刻进入。你可以心想我要入三禅,就专注于三禅的心念而入三禅,这时,绝对不是慢慢看呼吸地入定,而是念头在转、转、转,就转进去了。一旦你能够静下来心一想我要上三禅,心就这样转、转、转,转进三禅,你就有机会观察一个现象,经过几个心念就入了定。这在南传佛教的《清净道论》里有说到,从近行定入根本定的时候,一共应该经过几个心念的转变。如果你能以转变心念到某个禅定心念,如此直接入该禅定,以后你可以进一步训练超越禅,就是不按照次第入定,你可以训练自己直接进入三禅,再从三禅跳到初禅,从初禅跳到四禅,可以这样地跳来跳去。

当你把四禅八定都修完了,进一步,你要能够进第九个定——灭尽定。然而,不是每一个人修完了四禅,都能够进入灭尽定的。当一个人能入灭尽定,并且能从灭尽定出来时自由地跳到任何地定,如此定力称为狮子奋迅三昧。如果你只是在四禅里面跳来跳去,那只是超越禅。

灭尽定是一个很特别的定,进入灭尽定的人,如同死人一样,什么知觉都灭掉,什么心念也没有。由于无心,所以无法动念从灭尽定跳出来,必须在灭尽定的定力退了才能够出来。在灭尽定的定力刚刚退掉,就一跳跳到另一个定。灭尽定也叫灭受想定。没有想的定称为无心定。在无心定时,你不可能动念头想我要跳到三禅、跳到四禅。你只能在灭尽定刚一退出来的时候,一跳跳到三禅、跳到四禅。这是很艰难的事情,因为很少人能够入灭尽定。

在这里,我要求大家只修到四禅,不鼓励修无色界的定。为什么不主张修无色界的定呢?因为无色界的定力太强,太强的定力会妨碍你修观。在很强的定力时修观,一观就入定了,一入定你就不想动念修观了,没有观就不能发慧。所以外道修到无色界的定,整天定在里头出不来,根本无法修智慧观。因此,佛法说,一般人修到四禅的定力,是最好修观的时候。

\subsection{入定后的问题}

\subsubsection{境界的处理}

修禅定会出现很多问题,你们都知道如何处理。

修禅定的过程中,会出现许多境界。不管用什么样的修法,在禅定里产生任何境界,看到任何东西,要认定只是禅定的副产品,它不能帮你加强禅定,也不能帮你增加智慧。就是说,在禅定里面的任何境界,不过是一种神通,是禅定的副产品。你有兴趣就多看一点,可以增长见闻,但是看了就丢掉,千万不要将那些境界当作证悟,当作很了不起的事情。很多人在禅定里面遇到境界以后,会以为自己的境界比别人高。那是在比境界的高低,而不是比禅定的高低。他说他能看到佛,你能看到什么?他比这些东西,这些是没有用的。曾经某寺的一位法师向我修学禅定,他入定后看到很多佛菩萨围绕着,高兴得不得了;我觉得这没什么了不起。不要执着这些境界,这不过是副产品。千万要记住这句话。你看到佛会怎么样啊?就算佛在你的面前,你修不好也没有用,不如看自己的心最重要。你可以把境界当作游戏看看、听听,千万不要将它当作一件了不起的事情。如果你觉得了不起,你就要着魔了,肯定要着魔。为什么?魔最容易耍这种把戏给你看。所以,整个修的过程里面是要看心有多清净,有多定,这才是正确的修定的目的。

\subsubsection{气的处理}

在修禅定时气在身上运转,如果你不去注意它的话,那些气是不会很强的,如果你去注意它的话,气会因为你的注意而加强。有些人静坐时,任督脉的气会转动。如果你注意它转动,便会越来越强,后来变成在转周天。你可以去转周天,但是要有本事转。为什么呢?因为转周天转得好,会对身体非常好的,甚至很多病都消除了。但是,因为转周天的气很强,如果你没有一些气功的知识和基础,你转到一半转不了了,那就要出毛病。就是说在转周天的过程中,气跑到某个地方停下来,你不理它就休息了,致使很强的气留在身上某个部位,以后这个部位就要出毛病,那些转周天的人要小心气留在身上。如果你不去理会气,不去理会它的运转,气是不会很强的,就算它停在某个部位,也会很容易地自己化解掉。

另外,任何人静坐一段时间,身上、头上都会出现气。我们身上的气因为定力的变化而不同,心清净时,清净的气往上升。心淫欲时,浊气往下沉一直沉到男女根。当你心念很定很清净的时候,气就一直升、升,升到头上去。静坐完了,下座前要检查留在头上的气,若有的话要把它处理掉。如果天天累积气在头上,渐渐地会造成毛病,甚至头疼。

有些人,定力加强后,睡觉时也会入定,心定在一片光明里面,然后就睡着了。如果是这样,醒来的时候,头上有很多气,你必须去处理,不然以后也会造成毛病。

有些人,定力太强了,心一集中、看东西,就会入定。这些人专心看书或诵经的时候,因为心非常集中,外界的东西不能打扰他,很快就入定了。这些人有时头会重起来,那时就要小心注意是否气积在头上,是的话,要把气拉下来。

\subsubsection{其它修法}

有些人原本修过禅定,并且可以入定,我也可以接受你们用本来的修法。但是,你要懂得怎么用你原来的修法,一步步修上去,如果不懂,最好放弃原来的修法。还有,在禅定里面可以看到光明。有些人在四禅之后才看到光明,有些人在初禅就看到光明,看到光明时要小心,不要理会那光明。要回到应修的修法上,如果心跑去看那光明,就意味着你离开了呼吸的修法。可以用光明来入定吗?修定后见到光明可以放弃原来的修法吗?不是不可以,只是你要懂得以光明入更深的定。如果你放弃原来的修法,在光明里又不懂如何修,你就很难前进了。

\subsubsection{贪着乐受}

另一个会发生的事情,就是喜乐的感受。如果整天去观察喜乐的感受,就没法升更高的定。你要升更高的定,就要放弃原来的境界与原来的觉受。继续在修法上用功,你才可以前进。

\subsubsection{预感}

当你经常出进四禅,你的念力会特别强,也比较敏感,出现一些禅定的副产品——敏感与念力。比方说,他想到某件事情,过后那件事情就发生。你们可以去观察,但是,不要着迷在里面。会敏感到什么程度?比如在做事情的时候,你忽然想到某人,并且知道这是他在想你。这些忽然跑出的念头,一般人不去注意,它就过去了,修四禅者就敏感地觉察到。所以,一个经常入四禅的修道者,他就有能力知道徒弟发生了什么事情。听说过吗?徒弟不管在哪里,发生什么事情,某些师父都知道,这就是这位师父感应力特别强。这说明此人已成就四禅的功德,好多外道、内道修道者都能,别以为是修佛法有成就。

\subsubsection{念力}

四禅有很强的念力,比如,你心中想你少了一样东西,或者想要这样东西。不久就有人拿这样东西给你,修完四禅的人会发现经常发生这类事。或者,你想要找某某人,那个人就会来找你。再度声明:这些现象只是禅定的副产品。修禅定的目的是为了得到正念与清净心,明白吗?并不是为了得到神通。如果你有了这些能力,你可以帮助人家。但是千万不要整天玩这些玩意。为什么劝你们不要这样呢?因为你如果去注意这些事,而往往你身边就有太多的事让你感应了,你会因此忙得不得了。比如说,你现在在这里静坐,忽然感应某人在想什么,为何要感应他呢?太麻烦了。还有更敏感(个人隐私)的情况,就是平时对方想到和自己有关的事情,你都能感应到,其实那不是一件好事。人家会怕你。

\subsubsection{遥控治病}

另外,你修完四禅之后,可以学会知道别人是否入定。修完了四禅就有能力看某个人现在入什么定,这不是用眼睛看的,而是因心念微细,所以能感知他人是否入定。有了四禅的的念力,不用练气功也会气功了,自然能够发气收气。不只是如此,修完四禅后还可以用念力帮助别人治病,但得小心会因此把病气给惹上身。心念是没有时空限制的,所以,四禅的念力不但可以给人治病,还可以遥控治病。那些气功师能做到的,你也可以做到。这是念力的结果罢了。不信的话,你可以在这里给人发气,试试看给某处某人治病,这没有什么奥妙在里面。可是,我要警告你,如果你经常用念力去帮助人,你会越做越忙,因为很多人崇拜你。

今天就讲到这里,下一讲我要讲修观 。

\chapter{第四讲}

今天开始讲修观。

\section{见道并不难——不断贪嗔只破我见}

\subsection{何谓见道}

修观就是修慧的意思,修观为了生智慧,得心解脱。但是很多观法并非修慧,比如慈悲观并不是修观,而是一种修定的观想。我们要弄清楚,修观法门里,观想是修定,观察是修慧,修定清净心,修慧清净知见。哪一些观法是能产生智能的观法呢?佛法里有很多的修慧的观法,所观出来的智慧有深有浅。

\subsubsection{见惑与思惑}

解脱的次第是怎样的呢?部派佛教的论师把烦恼的解脱分为两大类,按照论典的说法,即见惑上的解脱,与思惑上的解脱。见惑是很多不正确的见解,使得我们不断地生死轮回。思惑是心理行为上的烦恼,有各种各样的贪、嗔、痴。

见惑与思惑是不一样的,这一点我们要弄清楚。见解的迷惑是你心中抱着一种看法、想法,然后就使得你做出许多错误的行为。其中最严重的错误是"我见"的见解,我见是可以非常快放下的。至于思惑是行为上的烦恼,就不可能一下子放下。我用一个比喻,比方说,你认为抽烟对身体好,这是一种见解。如果你对抽烟有贪着、喜欢,那不叫见解,那是行为烦恼中的贪。放下见解,是很快的,再也不会反复。再比方说,在清朝时期,中国女人缠脚,当时人们认为缠脚是漂亮的,那是一种见解。当大家都认为不漂亮的时候,女人也不再缠脚了。见惑的烦恼,只要看法一转变,就放下那个见解了。但是思惑的烦恼就不一样了。我们有各种贪、嗔、痴的心理行为,这类烦恼是要慢慢去改的。所以,在佛法中,把修行分为几个次第。先修资粮道,其次加行道,然后见道,修道,证道。

\subsubsection{资粮道}

就是你还不懂得修行前,先累积修行所要的资粮,做种种善法,积种种福德,多闻佛法学习戒律,这些都是资粮道。

\subsubsection{加行道}

就是你懂得根据佛法的道理,依法用功修行,这时以修定和修慧为主。这就是加行道。在加行道时修舍摩他和毗婆舍那,主要是让心生起五根五力,就是信、精进、念、定、慧。

\subsubsection{见道}

见道,见到什么道呢?就是见到解脱生死的道路。加行道用功修到有一定证悟时,见到解脱生死的道路要如何走。见到路不等于上路了,所以见道以后才上路——修道。也就是说,在你还没有见道以前,你是不知道怎么修道的。其实,修道是修心中的道路,并不是心外的道路。你要先看到心中的解脱道路,然后才决定在心中怎么走向解脱。见道就是知道心中的解脱道路。佛法说,当一个人见道的时候,他就破除了见解上的烦恼。那么,见惑要处理的烦恼最重要的是我见。我见不是我执,很多人搞不清楚。我们对"我"的执着有两方面。一个是行为上执着有我,一个是见解上执着有我。如果一个人见道了,就放下了见解上"有我"的执着。但是行为上还是执着"有我"。所以,当一个人见道——证初果,虽然思想上明知无我,但是,他的行为上还是贪生怕死,业习还是执着有我。初果的人还是怕死的,除非他是阿罗汉。我执在北传佛教分为两种:见解上"有我"的执着叫"分别我执",行为上执着"我"叫"俱生我执"。在南传佛教把分别我执称为我见,把俱生我执称为我慢。就是说见道了就放下了分别我执,但是,他过去业习带来的俱生我执还是有的。

\subsection{见道的条件——心清净与见清净}

要见道,要修什么法门呢?见道,并不是要先除去嗔心、也不是先除去贪心,而是先除去见解上"有我"的烦恼。因此,一个见道的人,他的贪、嗔、痴都在。贪、嗔、痴的习气不是见道所断除,这些习气要在见道后靠修道慢慢地改。所以,证了初果还是会贪吃、贪玩,但他懂得怎么去修行了。一定要非常清楚,见道只是见解上的烦恼处理干净了,不然的话,你会象那些没见道者,迷惑明明无我却为何还有我执。

见道需要什么条件呢?需要修一些``能破除我见的法门'',断我见的法门不一定是断贪嗔的法门。千万不要修错,是先修断我见的法门,不是先修断贪嗔的法门,如果倒过来修,是很难见道的,这就是为何好多人修几十年不能见道的原因。

学佛的人都听过佛法讲无我。但是根据佛法说,如果一个人还没有见道,虽然知道无我,他还会有怀疑的。他说:啊呀,虽然相信无我,但还是觉得有我,很多人都会这样的感受。但是,如果见道了,他就绝对肯定无我。无我并不是去找我在哪里,更不是找不到我在哪里而说无我。当一个人见道时,不会愚痴地去观察什么是我,他很清楚再也不需要找寻我,因为他已清楚看到因缘、因果现象。见道时必定看清楚,身心只是一系列的因缘变化,于是知道因缘中,我了不可得。

学过佛法的人,有一句口头禅"一切都是因缘"。那个人打我,那是过去世的因缘,这个人修行,他过去世有佛缘。就这样讲,但这是口说因缘法,他还没有实际地看到因缘法。不知道你们是否听过一个偈语:

若人生百岁

不见生灭法

不如生一日

而能得见知

这首偈语是说你活一百岁,没有见到生灭法,不如活一天而能见到。生灭法到处有,为什么百岁都难见?每个人天天都见因缘法嘛!不是这样子的,生灭法就是因缘变化过程,你要当下看清楚因缘变化过程,不是思惟。见道要快一点的话,必须修对方法,先不要管其它烦恼,只要管"我见"的烦恼就够了。因此,如果你选择正确,修对法门,就会很快见道。如果你修错了,你去处理嗔心呀,贪心呀,你在这一生都难见道,而且也不会相信他人能见道。为什么呢?因为处理贪、嗔需要很长的时间。甚至一个见了道的人,他的贪跟嗔可能今生都处理不完。但是,如果心清净时,要破除我见是非常快的,经典说不必几天就能见道,关键在哪里呢?佛法有这么一句话说:众生之所以佛法不能现前,是因为被烦恼遮盖住了。我们的烦恼并不是二十四小时都生起来的,至少你们现在在这里,好多烦恼没生起,就在这一刻大家都暂时没有嗔心、贪心。在烦恼没有生的时候,就有机会生起清净的心。佛说,如果一个人的心清净,佛法就能现前。就是说没有烦恼遮盖,心清净了,就有能力观察到本来现前的佛法。觉悟的第一步就是破除我见。见到佛法而觉悟了,并不等于所有的烦恼断了。

\subsection{破我见的捷径——观因缘最快}

为了破除我见,我们要观哪一些法呢?就是观因缘。因为``我见''就是迷惑、执着因缘里有个我的见解。既然要把因缘观察清楚,那么,我们要观那一些因缘?为什么一百岁都没有见到生灭法?应当观身心的因缘,不是观外在的因缘。哪一些因缘无须观察呢?如果你说我观察到种瓜得瓜,种豆得豆;这些因缘,农夫观了一生都无法觉悟无我。佛法告诉我们说,我们以为有真实存在的世间,其实并不实在。根据佛法说世间就是在你六根、六尘、六识里面作用。把六根、六尘、六识拿走了,就没有世间可说。六根、六尘、六识的作用统统都发生在你身心现前的当下。意思是说你是用眼睛、用耳朵、用六根来认识世间的境界。你要见道,就要在六根门头去观察究竟佛讲的因缘在哪里?观察这个``观察的行为'',不过是心法色法的作用,也就是六根、六尘、六识之间的种种前因后果。

这种种前因后果,很快的一个接着一个,快到我们来不及去思考。所以,要观察清楚,必须具备一个条件,就是你的心要很微细。心念微细就是要有相当的定力。这就是为何要你们先修好四禅再修观,当修到四禅的微细心念,就有能力以微细的心观察佛法。身心是快速无常的因缘变化,如果心念粗,根本来不及观,观察的心只能在因缘发生后,落入意识里思考。所以说:修观时只许观察,不允许思考!

\subsection{七觉支——觉悟时应生起的心}

佛法讲闻思修,思考是在思惟的阶段,修的时候是不思考的。为什么在修的时候不思考呢?思考都是学来的,在修的时候是观。观察是很微细的思惟,而且只是思惟单一事件。怎么思惟呢?修定时的思惟,如果观呼吸,什么也不用管,只做一件事情:观察呼吸。修慧时的思惟,在择法时什么佛法也不管,只管一句佛法。可能你们听过:一句佛法就够你证悟了。而且经典记载佛在指导众生证果的时候,都是一句话证果。那一句话是什么?它比什么都值钱啊。在道家里面有这样一句话:``假传万卷书,真传一句话。''听过吗?其实真传的那一句话,就在万卷书里面,没有离开万卷书。但是那一句话是在你刚好需要的时候,才告诉你——真传的那一句话。我们为了破除我执,所需要的佛法是观因果,你就专门观察六根门头的因果,其它佛法一概不理。佛说修不净观可以克服淫欲,修慈悲观可以克服嗔心,这些统统不管。你只管要观察的佛法,这个心态在佛法里面叫"择法"。就是说当一个人要觉悟的时候,他要具备一些条件,这些条件称为七觉支,如果你们念《阿弥陀经》会念到,七菩提分八圣道分。

\subsubsection{先要有正念}

这七觉支里面的第一个觉知就叫做念觉支。如果你想要觉悟,你必须要有很强的正念,这叫念觉支。然而,达到四禅的心念是最有效率的正念,若能一念之间入四禅,这样心力在修观时,正念最强最清净。正念用在哪里呢?就正念在你所修的法上。我们生起好的念头,做善的事情,这种正念是世间的正念,是不能解脱的。所以修观行时,并不是正念在做好事上。要想证悟、要解脱,是正念在所修的某句佛法上专心地观察。

\subsubsection{正确择法}

有了正念,接下来就是七觉支的第二个法——择法。这个择法就是你要选择所观察的法。不懂修行的人,他在修观时,思考一大堆佛法道理,这不叫择法,这叫思惟佛法。择法时专选一句佛法,任何时都不离开它地观察。比如说,我们在观察无常,对一切所对境都在观察无常,再也不作第二件事情,这才叫无常的择法。你可能说观察无常太容易:``你看!那里本来没有屋子的。后来,有人要建,建了屋子,这叫无常。''这绝对不是观察无常,也不叫择法。如果这是择法,世间人也会修了。择法是一种很专注、高度集中的心,选择某一句佛法来观察。择法时观察的对象是哪些法呢?择法的对象不是外面的境界,也不是内心的境界。是心和外境发生作用的当下,每个法究竟是怎么发生的,就是观察每一念当下究竟什么因缘在生灭。如果你要破我见——见道,不是观察外面有什么事情在发生,也不是在观察内心有什么烦恼发生。因为我们的心迷惑于境界,就应观察当下的心和境界之间有什么事情在发生。是观察能知心和所知境界的因缘,是观察内心和外境当下的前因后果。而这类因果是发生在很短的一刹那,并不是一个很长的故事,因为烦恼一刹那就过去了。所以,每当你择法,就在一刹间观察因缘、因果现象。其实,一切都是因果现象也,就是说你要在眼、耳等六根作用的一刹那,观察是什么因缘在作用。择法就是选择观``当下的因果现象'',只要你很清楚地在六根里面观察,就能把身心的因果作用看清楚。

\subsubsection{七觉支次第}

如果你有了四禅的清净的心,并以此正念再去做择法,你一定有能力看清楚佛法。当你看清楚之后就会很肯定:每一个心念和境界的作用,它们从哪里来,怎么发生,我都能够看到。如果你能够看到每一刹那的念头和境界之间的因缘,这时你的心必然会很微细。因为择法让你进入很微细的心,你越看越清楚,越清楚于是越想看,于是生起精进心。看清楚佛法会使你全身兴奋,你会越看越兴奋而觉得头发涨,充满气,这时便生起法喜的心。由于法喜充满,身心进一步轻安。因此择法而越看越集中,慢慢地,心念越看越平静而生起定心。内心次第生起念觉支,择法觉支、精进觉支、喜觉支、轻安觉支、定觉支。最后,你会进入没有造作的心态,面对任何境界,内心只观察因缘,心不再动念反应、没有造作了,这叫舍心。七觉支会一个接着一个地生起来,当七觉支一个个生起来,你就会越修越有信心。你还会发现,你择法越清楚,烦恼越轻,身心越愉快。如果你修错了,你的心就会越观越沉重,越观越烦。你修对了,越观越欢喜,越观越快乐。这是修法的欢喜。法喜充满就从这里来。

\subsubsection{破我见的择法}

当修观在择法时,绝对不允许去思惟任何其它佛法道理,只能够观察所择的法。择法是要观察到当下事情的真相,思惟是观察以后的事情,这一点要分辨清楚。

如果你很仔细地观察身心的因果变化,你会知道你就是在看清楚当下,心中不思维任何道理,对所观的法犹如见亲爹娘那么自信,不须依道理证明。如果你完成了这一点,你就会肯定没有我了。而且,你会肯定地跟自己说:``是我亲自看到一切只是因缘作用而已,身心就是这么回事,不是道理上无我,真的是本来无我。''你会自己确定:``虽然无我,只要有因缘,就会说话、会走路、会造业、会起烦恼。''你看到的都是一系列的因缘变化,一切都无我地在运作,于是,你便见道了。

如果你不懂在哪里择法的话,你可能去观察:``我现在有没有起嗔心啊?现在我有没有起贪心啊?''如果去观察这些,是不可能很快见道的。为什么呢?因为处理贪嗔是要花很长时间的,那是见道以后,修道时才修的法。见道是一见便永不再迷惑,行为上的烦恼习惯不是一刹那就能彻底断除的,所以,修道要慢慢修,慢慢断烦恼,修道要长久修下去。我如此解析,是想让你明白,如何选择正确的修法来破除我见。在修行的次第里,你必须要有清净的心,才能够去观察佛法。经典里面说,必须先要心清净,然后,才能见清净。所谓见清净,就是生起正见,破除我见。当一个人观察到身心只是因果作用,自然地就知道身心无我,也知道戒律的因果作用,他不会随便乱持戒律的。见道者自然对自己所修的法不再怀疑,并且知道就这样一直用功修下去,终归能够解脱。也就是说他已清楚地看到了那道路,并且心中朝向着那条解脱之路。

\subsection{见道的难易}

\subsubsection{证初果的难处}

一个人见道之后,贪、嗔、痴还是一大堆的。为什么呢?因为贪、嗔、痴才是修道所要断的烦恼。要证初果,关键在于有没有生起清净的心和是否正确择法。如果你择法时想怎么去行菩萨道,你就不可能很快见道。为什么呢?菩萨道是长远的,不是一下做到的,为了行菩萨道,心整天向外做利益众生的事,没有去观察自己内心,就难见道了。所以,菩萨道要修很长久的时间。但是如果要见道,却不是很长久的事情。因此,我们对证初果要有信心,要弄清楚证初果难在什么地方,它的难处在于以下几点:

1、你认识第四禅舍念清净地的舍念与清净心吗?有没有以清净的心来观察佛法?

2、在修观时有没有放下一切的思惟、一切修法,只选择要修的法来观?

3、多数人不懂得如何择法,所观的法是清理贪嗔呢还是清理我见?如果是清理贪嗔,难矣!

4、对证初果没有信心,就不可能见道。当你见道了,没有信心者总会怀疑你在大妄语。

\subsubsection{末法还能证初果吗}

很多人不敢相信这个时代还能够证初果。我用数字来使你们生起信心:佛两千五百年前度众生。当时,度了好几万的阿罗汉,身边常随众就有一千两百五十人。阿罗汉那么多,三果就更多。二果、初果就更不用说了。佛在世时初果有多少人?可以这样估计,阿罗汉至少上万,初果至少几十万啊!依佛法说这几十万人,证初果后还要继续修行的。佛灭后,那些只证初果的人还要继续修,不是到天上就回来人间。也就是说这些初果的人要陆陆续续回人间。就以两千五百年来除二十五万吧,两千五百年每年多少人得道?也就是说每年至少有几千人来证果。还有那些要证还没证的呢。他也要来嘛。所以,就算是末法,也会有人继续修行,乃至自己证果。他证悟了,也没人知道。不要听人家说:现在末法了,是不可能证果的。其实你用数字去衡量,就可以知道证果是可能的。

\subsubsection{假传万卷书,真传一句话}

谈到证果的问题,中国佛教界有个很矛盾的行为。大家想要修行解脱,讲经说法的法师也劝人要解脱,却不信今生修行能够解脱。南传佛教修行人都相信能证果,有很多法师和讲经师听了不信,认为这个时代怎么可能,证果的人肯定是个疯子。但是,我们要有自信,不然佛法不灵验了。在中国佛教还有一个更糟的现象,一听说有人修禅定到了四禅,都说那是大妄语,尤其佛学院都说不可能,修净土的人更不用说,这些都是没有遇明师指点,所以认为是很难的。

\subsubsection{根据经典里的道理去修就可以吗?}

可以!但是,你不懂得大藏经里,哪些经典是你目前最需要的。刚才说过:假传万卷书,真传一句话。那句话为何不能写在经书里呢?那句话在道家叫"火候",火候是无法靠文字传的,唯有靠经验过者,在火候到时告诉你,就是这个。你修到的那个工夫的当时,只需要某句佛法,老师当时一说你就领悟。如果当时把万卷书给你,你就被误导了。修行的火候就是这样,遇到一个明眼人告诉你,你目前就需要观这句佛法,就够你受用了。其它的法,你可以暂时放在一边。多数人不依老师指导,以为把经书背得滚瓜烂熟就会修行,其实那是盲目依某部经典去盲修,于是东修一点西修一点,这类修行人就算修到有成就了,也不知道自己是如何修成的。

\subsubsection{不信者难以依法入定与证悟}

我所教的这系列修法,虽在南传佛教流传,当地有些人却不以为然。其实,自古以来,很多修行成就者,都有很多人不同意它。你们看过六祖坛经吗?六祖在五祖指导下证悟了,没人信他,认为像他这种人怎么可能证悟。一个人能不能证悟,不能看外表,也不能看学问,证悟跟学问绝对无关,是和他的定力、他的智慧和烦恼有关。绝对不是会因为背书就有智慧的。

对证果的怀疑,是造成你不能证果的最关键原因。疑的烦恼会障碍修行。如果你怀疑某个法门,就没法修好。如果你怀疑这个法门的老师,你也没办法修好。怀疑自己不能修,也不能修好。所以,今天第一件要事,就是告诉你们,在这个时代还是能证果的!我不厌其烦地要你对自己生起信心。我很清楚当我这样讲,就会有人骂我狂言惑众。可是,我非常自信地认为,那些相信能证果的人必会得益,因为佛说若有人信佛、信法、信僧、信戒,有此四不坏信者就是初果,你有此四不坏信吗?若有,剩下的事,只欠去认识修法和去认识初果而已。为何要听从无四不坏信者说的话呢!决无此道理!

\subsubsection{无智人前莫说,打你色身星散}

这句话是永嘉大师证悟后说的,意思是勿让人知你的证悟。若你对此四不坏信有信心,当你依法修了有所证悟,不要随便跟人说。第一点,无四不坏信者不信你能。第二点,你未必讲得赢他,理论讲输了就是你错。所以,不要随便让人家知道。还有,如果,你整天跟人家讲:我修得怎么样怎么样,你又犯上了我慢,怕人家不知道你有修行。你不是阿罗汉就一定有我慢,所以我慢是很正常的。但是,每当你这样讲,你就一直在增加我慢。关于修行的成就,没有所谓老师一定比徒弟厉害。只是谁先谁后的问题,只是老师先学会,有缘由老师告诉你,仅此而已。可能你修得比老师好都不一定,不要以为老师绝对比学生厉害。假设你是三地的菩萨来人间投胎时,我是一个老和尚,你来跟我这个老和尚出家。当然是跟我学嘛,对不对?虽然老和尚是个凡夫,他先懂得佛法,菩萨于是向他学法。所以说,修行嘛,三地菩萨的老师不一定是四地菩萨。这个比喻告诉你,不要以为你一定比老师差,老师也不一定胜过你。各有因缘,各自成就自己的道果。

\subsubsection{修观修错变成修定}

在修观的时候,定力太强观行就修不好。前面说过,在四禅八定里的五、六、七、八的定很难修观。如果一个人的定力非常强,他在修观的时候,一下子就入定了,他变成修定还以为自己在修观,结果他在浪费时间。往往很多人不明白,他在那里修啊,修啊,修到入定了,他还以为他修得很好。如果你修观修到入定,你就修错了。

所以修观的时候,不允许你们入定太深,但是又不能没有定。因此,在修观的时候要天天保持四禅的定力,就是说,你一定要做到每天有能力上四禅,而且一下子就上。但是修观时,不可以入四禅太久,如果你每天入四禅太久,就修不了观了。为什么呢?因为定的力量使你的心不想动,于是观没两下子就入了定,那样你每天都在浪费时间。别人却以为你很用功修道,而且道行不得了,整天坐在那里动也不动。但是,在我看来你修错了。所以,当你修完定以后,我不会鼓励你们入定入太久。如果你们想要修神通,你们就应该长久入定;长久入定就能引发神通。如果你修观,观到很仔细,很清楚以后,你会觉得入定是很无聊的事。为什么?进去了,终归要出来,出来了,又想要进去。所以,别忘了修定的目的是为修慧,为了修慧而训练不同的心力——定力。

你们明白了这些,以后修观就懂得怎么做。

\subsubsection{择法重点:除所择之法,不许思惟其它佛法}

修观最重要的重点就是你一定要保持心念清净。修观时跟修定一样,不允许动第二个杂念。我们在修定观呼吸时,不允许动任何杂念,动了以后,你就要从禅定掉下来。在修观做择法时,只允许一个时期观察一句佛法,不可以动第二个佛法的念头,要如此修。如果你懂得正确的择法,七觉支很快就会生起来。如果你修一段时间,就来向我报告你的觉悟,我会当你是放屁打妄想,因为修观时去玩味所觉悟的道理是没有用处的。为什么这样说?你不知道是在打妄想,还以为自己有所领悟。在整个修观的过程里面,你要得到的是什么呢?在修观时,觉悟不在所观的道理上,而是生起七觉支:念觉支、择法觉支、精进觉支、轻安觉支、喜觉支、定觉支、舍觉支,让这七种心念生起来,就能觉悟了。所以,修观时生起任何其它佛法知见都是妄想,要立刻斩掉它。懂得修出七觉支,你就会很快证果。关于修观,今天就讲到这里。

\chapter{第五讲}

\section{修止观常见的问题}

由于大家修禅定时,还是犯上许多错误,今天继续讲一些关于我们静坐时发生的问题。

\subsection{不可引气入定}

先说关于头上有气。

在修禅定的过程中,很多人都会觉得头上有气,知道有气的比不知道好。可是有些人以为头上有气就是入了禅定,这是错误的。入定的确使气升到头上去,但头上有气不等于已经入定。就是说,如果你故意把气引到头上去,是气集中在头上,不等于入定。这一点要非常小心,不要把气带到头上去,然后以为自己上三禅了,那是无效的。

入禅定是靠心念的集中和微细而入,不是靠头上的气上上下下。所以,你觉得头上有气,先把它引导下来,然后再度集中看呼吸。要注意是呼吸及心念变细了,才上初禅去,不是把气带上头去就以为上禅定,因为懂得气功的人就懂得只要我心念集中,就能把气带到头顶上去了。甚至你心想身上的气往头上冲,气就冲上去了。所以,不要犯这个毛病,以为气在头顶,就上了三禅,然后在那里面呆啊呆,没有用的。要心的杂念少了,集中力够了,轻安来了,那时想上禅定就自然上去,那才是对的,千万要注意这一点。

\subsection{疼痛的问题}

另外就是疼痛的问题。有些人在静坐的时候,身上有疼痛,不少人以为疼痛是打坐出了问题,其实这不是修错,是那部位有障碍,有暗病发作了。最好将暗病处理掉,如果让它继续保留在那里,你就一直坐不好。比如你静坐的时候,觉得胸闷、咳嗽,你就要处理胸前的毛病。如果你觉得背后大椎以下很热,就是你的大椎的部位阻塞了,你最好去将它通顺。如果你的颈项这里很热,就是玉枕穴的部位气不通,阻塞了。简单的处理法就是用空心掌打拍,也可以找人引导,在你静坐时帮你将气引上去,疏通它。如果你自己懂得做的话,你就在静坐时,观想背后的气从大椎经过玉枕,一直这样想,慢慢地气通了,热就会消掉。大椎和玉枕是很关键的部位,要去注意有无阻塞,如果很热的话,就是有问题了。

\subsection{身体摆动的问题}

有些人因为身上的气脉有阻塞,坐到一定程度,身体会动起来。如果身体要动,不要去阻止它。你越阻止它,就越坐得不好。这是气动,气动会越动越强,如果心力和念力不够,当气很强时,心力弱者若无法控制气,就会有危险。当气不受控制而身体出大动作,一般人就会慌张,慌张最容易着魔。将来若发生这类情况时心不要慌,只要一心想全身的气慢慢地降回丹田。并将所有的气,慢慢沉回丹田,一直这样重复地想。如果你的心念足够的话,你可以用手来回从头向丹田拉气,观想手带动气到丹田,最后按住丹田。如果做了气还是在头上,可以找其它人帮他一下,怎么帮?就是用掌心将他的气,慢慢地从他的头上引导回丹田。也可以用拍打,帮他把身上的气拍打散掉。

是什么原因造成动得很厉害呢?原来气动是在治病,所以会觉得越动越舒服。身体越动得久,气场就会越强。如果你们以后身体动起来,千万不要动超过半小时。若超过半小时,就会有很强的气,容易不受控制。动得很强时,怎么处理呢?我们可以控制动的时间,就是说,先让他动几分钟,然后就心想气收回来,让身体不动。先只动五分钟,多做几次后,如果都有把握把气收回来,就可以每次动十分钟。这样慢慢地加强,以后,你就会有能力收。不要让它一次动得太厉害,收不了不是着魔,是因为你的念力不够。什么情况下会着魔呢?当身体动时,有些人会气胀脑昏,身体就会失控而乱动,就会着魔,会发狂、发生精神错乱。所以,动时要小心,心念要很清醒,心要明白动是好事,只是身上的气在运转。不要恐吓自己以为着魔,那样就会有危险,所以你应当放心让它动。气动而身动,这会自动调整身体,你身上的一些障碍就会驱除掉,以后你的静坐就会更有进展。因为你身上有气的障碍,才会发生身动。如果你忍住不动,气的障碍依旧,静坐的进度就会很慢。若有气阻塞,最好找一些气功师,帮助你处理那些气阻塞的地方,或者找人按摩等等。身会动的人如果忍住不让身体动,不动则气不通,静坐很难有进步,因为严重气阻塞才会动,所以要小心处理。动的时候,心念要明明了了,很清楚,心中要注意那个气的运转,不能什么也不管。练气功的人喜欢让身体动,动时心里什么也不管,这个是有危险的。你的心念一定要清楚地在气动,要防止它动的太厉害,才不会有危险。以上是静坐时,有关气动的一些危险的事情。

\subsection{下坐之前按摩的重要性}

每次静坐下座之前,全身都要按摩一下。这几天我看到大多数人都没有按摩,久了要出事。就是说,下座时要保证气收回丹田,眼睛未睁开前,手擦热后,用手掌按摩头,拍打头也可以。然后,掌心向着自己,观想把那气引导到丹田来,从头往丹田引导。除非你对气很敏感,很清楚知道气都收回了,你可以不用这么做。如果收不回,你一定要做一下气收回丹田的动作。如果你对气不敏感,每天留一点气在头上,连续几天以后,就难处理了,从此以后,每次静坐头就会有重重的感觉。所以,最好坐完后按摩。还有,如果每天静坐六小时以上,因为长期保持一个姿势,身体也会产生疲劳,最好每次起身之后做一些运动,放松后背及腿的肌肉,对你继续坐会有帮助。如果你没有这样做的话。那疲劳就要累积,一段时间之后,你再坐,就会不舒服。以上所说的,是静坐上要注意的一些问题。

\subsection{修定与念佛}

修不好禅定不要心急,每个人的过去世和今生的修行不一样。有些人前世已经修了今生又来修,有些人是原本什么定力都没有就来修的。因此,不要跟人家比较,如果你坐来坐去都不行,就是你在修定方面的工夫不够。你更应该努力用功,不要说我无法修定,还是去念佛好了,这是错误的。就是说,你原本定力不好,就更需要修定。不要因为修不好定,就不修定了。你以为去念佛,就不必修定了吗?不是这样的。佛教的修行要修戒、定、慧。念佛是一个方便法门,很多修念佛法门的人,念很多年了,他都没有一定把握的。不要以为念佛的人都是有把握往生净土的。我讲一个例子。在我们那里有一个老和尚,叫云水山人。他满身臭气去挂单,大家都不喜欢他。大概在他六十岁的时候,就自己做经忏,搞了一个房子住,后来他病倒了,没有人照顾,我就找了一些居士照顾他。他念佛念了二十年,到老的时候,他更加用功念。但是当他病了住在医院里的时候,他就埋怨说:``我念佛念了二十多年,现在这样病重,阿弥陀佛还不来接引我。''你不要以为念佛很容易,不是的。那位老和尚到后来根本就不念了,不信了,他说阿弥陀佛根本就没有来接引。其实,这是错误的。经典上说,是你死的那一刻阿弥陀佛来接引,不是你活的那一刻。佛不会带你去死亡。另外,在台湾有一位修念佛法门的大居士,他有两个妻子,他特别疼爱他的小妾。他念佛念到预知时至,就要大家为他助念,念了几天后,这件事情就传开了,传到了他的妾那里,他的妾很爱他,听到他要走了,就很慌忙,打电话来找他,他听了小妾的电话之后,就念不了佛啦,然后,就担心他的妾,想我走了,她可怎么办?哈哈。

就是说,如果念佛的人对世间还有贪爱的东西,临死的时候是走不了的。这是很重要的一件事情。但是,很多念佛的人不知道自己还在贪爱人世间,真正死亡境界来时才发现自己放不下,然而已经太迟了,比如刚才说的那位居士念佛念到要走时他都知道了。可是后来呢?还好,他还够智慧,干脆就不理他的妾,避而不见,专心念佛,后来才走。你看,念佛往生容易不容易啊?别忘了,临终时若无定力者,心乱如麻,如何能往生呢!念佛念到一定程度是要产生定的。真正念佛法门是能够进入念佛三昧。念佛、观想佛者要是能入定,这样你临终的时候,就因有定力而有把握往生。如果你只是持名念佛没有定力,就要靠别人来助念。修净土的人若执着身边的事物,那是往生不了的。所以说有了定,往生净土就多了一分把握。

\section{如何修道}

先见道后才修道,整个过程是从内心建立三十七道品,其中,七菩提分以前未见道,七菩提分时见道,七菩提分后是修道。

四念处——资量道,出世正见。

四正勤——资量道,灭恶生善。

四神足——资量道,净心得通。

五根——加行道。

五力——加行道。

七觉支——见道。

八正道——修道。

\subsection{解脱的起点——五根、五力}

佛教导我们修任何法门,都要修戒、定、慧。就算是你往生极乐世界,到那边学哪些佛法呢?如果你们有念《佛说阿弥陀经》就知道,那边还是听闻五根、五力、七菩提分、八圣道分。真正的解脱法就是这些佛法,到了极乐世界还是这些佛法。为什么是这样的呢?因为修行的过程,并非让心充满真理,而是让心建立起五根、五力、七菩提分、八圣道分,这些佛法是不可能靠外力加持的。所谓的心的道路,修任何法门都是如此,诸佛皆说此法,但是很多人不明白,以为这是小乘法。其实五根、五力、七菩提分、八圣道分,是你修道时心中必须生起的各种各样的法。

就以五根来说,五根是信根、念根、精进根、慧根、定根。信根是你对佛法要生起信心,一定要有的。精进也是一定要有的。精进不是努力地去做事情,也不是努力地去修的意思。精进是一种心念,称为精进心所。不是说我要很努力地去做一种事情就有精进心所,精进心所是修道时才产生,当精进心所生起来时,你就会很努力地去修。精进并非有一件事情会让你很有兴趣去做,它是修道时有精进的心念,叫做精进根。所以,这些各种的根:信根、念根、慧根、精进根、定根,都是从心念上建立修道的力量。也就是说,你修行时要建立起这些法。这些法加强了就是力,叫五力。五力进一步加强就是菩提分,再修就是八圣道。所以,不管你到哪里,不管你修任何法门,你都要在内心建立起这些法,这些法是任何修法的``道品''。修道并非从心外得到什么法让你不得了,就算你到极乐世界,也要从自心建立起这些道品。为什么你不要在这里建立?应该在这里训练嘛。到了极乐世界,你就会快一点嘛。所以,我们如今在这里修,就是要训练这些心念。

前面,有讲到七菩提分,其实,如果按照修道次第,加行道就是建立五根、五力,加行道绝对不是象一般修法的加行法门。比如说大圆满修法,有前行、加行、正行。藏密修法的加行是指定要修某些法,这些加行法不同于加行道的加行,任何修法的真正加行道就是要建立起五根五力。任何的修法,如果进入加行道的状态,就是心中具足五根五力。我们修观的过程,也就是在建立五根五力。

\subsection{四念处}

如果是根据小乘的佛法说,你要建立五根五力,要先从四念处着手,为什么要从四念处着手呢?这是有它的道理的,四念处的第一个念处是身念处,然后是受念处,心念处,法念处。这样的次第是跟心念的粗细有关,就是说身念处是很粗的,心念很粗的人都可以观察。法念处是很细的,要心念很微细的人才可以观察。初学者一定先从粗的心和境界开始观察,如果身体的动作都没有办法观察清楚,你想观察清楚心念那是不可能的。所以说,一个人能很细心的观心念处的话,他的身念处必然会做得很好。因此,念处的身、受、心、法的次第是绝对的。就是说从粗到细,当然,如果你有本事,你可以一下子去做细的,不必要先做粗的。

\subsubsection{粗的观——身念处}

在观察身念处时,我叫你们经行注意脚,也要注意避免心爱听爱看而离开身念处;还有要注意你身上的任何动作,这些动作都是属于身念处。就是你对身上的一切动作、行为明明了了。身念处从什么时候修起呢?很多人从站起来经行时开始修身念处,那就错了。是从你静坐想起身的时候开始修。从坐禅下坐开始,对于你的一切行、住、坐、卧,都要很冷静在那里修身念处。修得好时,你不会东张西望,不会想要和人家谈话的。如果你没有修好呢,就会等到想经行时,摆好姿势才修身念处,那样修身念处就太差了。

你们经行时要注意这一点,就是说,从你下坐不再盘腿,脚一举一动都要清清楚楚。如果你懂得修,很用功修的话,经行时也是在修定,如果你不知道静坐与经行交替修,你就无法长时间修定,只能静坐后休息,如果你懂得修的话,你就随时随刻都在保持正念,跟静坐的意义是一样的。

\subsubsection{细的观——受念处}

当你在观察受念处的时候,要分别出有苦受,有乐受,有不苦不乐受,进一步观察这三种受都是苦。当乐受来的时候,你要观察,有没有起贪爱的心,苦受来的时候有没有起嗔心。如果有,你要将这些贪心和嗔心放下。你们在平常生活中要多观察受念处,可是在闭关的环境,是没有必要观察受的。为什么呢?因为在一期的修道和闭关时,修观受是难以成就的,因为在一切已经准备好的环境,没有人来干扰你,当时心情愉快地经行,没有苦受和乐受,多数是一种不苦不乐受,此时少贪少嗔,没有机会处理贪嗔。只有你在平时做事情,在生活中来观察受,才有机会把受之后生起的贪心和嗔心放下。所以,在安排好的修行环境里,很难有机会修好受念处,唯有在平时做事情的时候,才有办法修好。观受是苦是受念处的主要目的,但是,不一定都要用这个修法来观苦,你们修观的任何方法,也能观到苦。其实观心、观因缘等也可以观到苦,不一定要观受是苦。如果你懂得观的话,一切无常生灭的法都是苦。

如果你修观修到心清净之后,肯定会观察到苦,没有观到苦,是你的智慧还没有生起来。我说过,当你在修观的时候,一旦七觉支生起来,会越修心情越快乐。虽然如此,还是有一些观法会让你的心很苦。我说过,如果你想要证初果,重点是放下我见,不是去除贪、嗔、痴。因此,你要选择容易让你处理我见的观法,那你就修得快。如果,你去选择比较久的修法,我用个比喻,如果你嗔心很重,你去观呼吸,观来观去,嗔心还是很重,观呼吸并不是帮助你解决嗔心的修法,要克服嗔心就应该修慈心观。同样地,观受是苦也不能帮助你放下我见。你越观受是苦,越让你生起出离心而已。所以,要破除我见,是观因缘、观因果。如果你要去除贪跟嗔,而我们贪、嗔的烦恼很多,其中有两个修法,一定要修,一个是慈心观,一个是不净观。这个贪跟嗔不是见解,它是一种业习行为。慢慢地放下。所以说,见道以后修道。而见道以后修道是为了除掉贪、嗔。重点在贪、嗔。

\subsubsection{更细的观——心念处}

至于说心念处的观心无常,我们在修观的时候能够观到,当你很清楚看见自己的心念之后,我会叫你们尽量地观察内心。同样地是观察内心,跟禅定一样,也有境界的粗细。一禅、二禅、三禅、四禅,有粗有细。心是念念在无常生灭。这个心无常的生灭,可以观到很细很细的生灭,也可以观到很粗很粗的生灭。如果在很粗的心里观,你很容易打妄想,你就观不清楚。所以,我们观心的方法也要从粗的观法进入到微细的观法。因此,在观完因果的时候,心念处要从观五遍行修起。

\subsection{心念之头——心念处的五遍行观}

五遍行就是触、作意、受、想、思,这五个心念是我们内心接触境界时的第一念。心没有第一念,第一念是根据境界来说的,随着境界出现所引起的第一个念叫某境界的第一念。在任何境界的第一念就是遍行心所,一共有五个心所,又叫做遍行五。一定要把它看清楚。但是,每一个境界现前的那一刹那,当时的第一个念就是五遍行。就是触、作意、受、想、思,要观心念生灭的人,最少要有能力看到五遍行。为什么?你们知道参禅有参话头。话头就是话之头,话之头是什么?就是心念嘛!还没有开始说话,要动念之前。动念之前五遍行已经先动了。第一念没办法看回自己。一定是第二念看第一念,也就是说,你持续在第二念观察第一念,就是观第一念了。当你能够掌握心念一直保持观第一念的时候,你才能修观心无常。如果,你没办法看到五遍行,你不可能说观心无常。为什么?烦恼来后才观,那是观法尘非观心了。所以,你要修心念处,观心无常要观察到起心动念的第一念,就必须把五遍行修好来。任何人修行停了一段时间后,再想继续观心念,也一样地要先观回五遍行,再度将五遍行观清楚了,才来观心无常。如果没有办法掌握到第一念的五遍行,他很难观心无常。

放下我见之后,所要清理的烦恼,是欲界贪、嗔和五上分结。我们的烦恼太多了,不可能把各种法门都修好。最快捷的方法就是把心念看清楚了,这样来清理烦恼最快。更快的就是觉知的心跑到烦恼的前头来清理烦恼,当烦恼一生起来你就知道,这样清理烦恼最快。所以说见道以后才懂得修道,因为见道的人能够看清楚心念,他懂得如何在心中的道路上,观察及处理烦恼。当我们在观察心念时,必须越观察越微细。微细到觉知的心跑到烦恼的前头。所以要训练让你的觉知心跑到你的烦恼的前头。如果你要完成这一点,就要先训练看到第一念。

那么,这时如何观心?心念有很多很多种,太多了。妄想杂念不是所要观的心,那太多太多的念头不是你能知的心。我们的心分为能知和所知,所知的念头称为心所,能知的称为心王,最简单的说法就是意识和法尘。所谓观心无常,并不是观法尘无常。也不是观察想东想西的心无常,更不是观受苦受乐这个心无常。如果你观这些心就修错了,这些是粗的观心无常。观心无常绝对不是心里面有一些念头让你知道,然后你观它无常。观心无常是什么念想也不要去理会,只观能知的心识在里面跳动,注意!心识不是念头!这才是观心无常。可能有一些人不明白,我就举一个例子。比方说,你们念南无阿弥陀佛,一般念佛者说念南无阿弥陀佛的是我的心,其实(南无阿弥陀佛是六个想)想是法尘,受也是法尘,知道想和受的知觉才是应观的心。当你在念佛号时其实是想佛号,念南无阿弥陀佛不只是一个想,是六个想。六字有六个想,每一个想,都有一念心在知道它。每一个知道想的心,是一念心识。也就是说,你心中念"南"的时候,心知道你在念"南",知道就是意识,所知道的是南的念头,即法尘。你念"无"的时候,心知道念"无",这又是另外一念的意识。这一念一念的意识随着法尘生起来。观心无常不是观法尘,观心不是观法尘的南无阿弥陀佛的念想,是观那些知道念南无阿弥陀佛的心念。如果你没有弄清楚,你说你在观心无常,其实是观法尘无常而已,这一点要弄得清清楚楚。参话头也是一样的。当我们在观心念生灭的时候,你的心就要远离外境,只观自己的意识,观到最后,你的心就会非常微细,最后到无念。

禅宗说一个人参禅参到一定的时候,会桶底脱落。桶底脱落就是进入一个状态,心中什么东西也没有了。空空洞洞但是心中又明明了了。我们在观心念的时候,也能够完成这样的事情。心中什么事情也没有,但是明明了了,一个念头也不动。从观心念生灭进入无念是慢慢进入的,然后要有办法保持。在禅宗里面说:"无心犹隔一重关"。就是无心还不能解脱。当你能够做到无心的时候,你就明白要修行快一点,要快一点解脱,就是平时必须要保持无心的状态。修行到这种程度,再也不要在座上修观,就要平时对境修,任何贪、嗔、痴生起来,心一觉察到就把它丢掉。因为能观的心已经站在烦恼的前面,烦恼心一动立刻就知道了,立刻就放下了。这样来克服烦恼是最快的,虽然快,烦恼也不是一下子就可以清理完。就是说,看你烦恼有多少,而且有时候,你没有面对它,你就不知道。你如果能做到无心以后,你再也不能躲起来修,你要去面对境界了。为什么?因为你躲起来修,就没有任何让你生起各种烦恼的机会,你也就根本没有机会去处理烦恼。

\subsection{身心苦与出离——法念处的五蕴观}

当你能掌握第一念之后,就有能力观好五蕴。如果你们看过心经,你们就会发现很多人不知道佛法的重点在哪里。佛法的重点其实就是《心经》所讲的,就是五蕴十八界、十二因缘、四圣谛这些法皆空无自性,不但是佛法的精华,整个般若的精华都在这里。但是,很多大乘的法师认为这是小乘的法,我们要发无量广大的心,我们要象观音菩萨那样。其实,所谓你的身心和你的世界,就在五蕴十八界里。不要以为世界是一回事,五蕴身心是一回事,绝对不是,你的身心就是五蕴,世间就是十八界。你的五蕴身心在十八界里面生活,但是你以为自己在世间生活。这怎么说呢?你说我在看东西。其实不是。是眼根接处色尘生起眼识,是这三样事情在作用。但是,你说我看到东西了。佛法说绝对不是。为什么说不是呢?因为,不同的眼根看同一个境界是不同的结果。你们听说过吗?不同业道的众生,因眼根的不同,看同样的东西结果都不一样。你以为外面的一切法并不在外面,绝对是十八界告诉你的。你说你要修观,观什么?观外境?观内心?其实都是观五蕴、十八界。

实际上,任何高深法门,在修观转迷成悟时,都是观五蕴十八界——观心。你要从五蕴十八界观察到其中的苦、空、无常、无我,去放下烦恼、执着,这才是修行所要面对所要观察的法。所以,《心经》里面提到的佛法精华,就是在五蕴十八界里面证悟空性。并不是观五蕴十八界以外有这个闹钟(指讲台上的钟),然后观它本身没有自性、是空的。绝对不是,为什么?当你认为五蕴十八界以外有一个闹钟,你已经被五蕴所骗,你还在自欺欺人地说,你知道闹钟是空的。你不知道那不过是眼根和色尘的因缘现象而已,你还说你知道任何东西是性空的,然后你在那里观空,那是迷上加迷。所以,一切佛法修行,最后都要观回自己的五蕴十八界。没有外在的什么佛法道理,因为你所知的一切都发生在五蕴十八界里,你还以为有外在的。所以,在修观的过程里面,要你们观五蕴十八界,把它看清楚。当你看清楚的时候,你会发现一件事情,你的五蕴身心活动一直在逼你前进。没有一件事情是你能够做得了主的,没有一件事情是你要的。绝对要看到这回事,你才会知道哪有什么外在的东西可爱可恨,统统都是自己内在的东西。你才懂得怎么样去放下贪、嗔、痴。那么,你就再也不会想要那些虚幻的外在事物,不想再被它迷惑,出离的心就会殷重起来。这时的你就会知道不是出离这个世间,而是出离五蕴十八界,不再迷惑于五蕴,不再以为有外在的法。不懂的人以为有外在的世间,我要跳出三界,却不知三界就在五蕴十八界里面。

\subsection{烦恼之根——法念处的心念生灭观}

如果你们继续看下去的话,接下来,你必须进入非常微细的心,观更微细的心念生灭。任何修法,只要越修心念越细,正念越清楚,就是正确的修法。如果越修心念不是越细,以为越修就能得到越多的法,那都是错的。世间有什么东西给你利益啊?世间就是五蕴十八界的把戏嘛,不懂的人会以为修行能得到什么了不起的法。所以,观心念的生灭,必须越观心念越来越微细,看清一切不过是心经所说的佛法精华——五蕴皆空,除此别无更殊胜的法可修可观,若有,必定是观心中之法。

以前说过,我们在修行的过程里面,修观不同于修定。任何人修定的时候,因为心念的专注,会把能量集中在身上。头上会累积了一股气,会有一股气包住头,修观时头顶会有气往上冒。当你观的心越细,因为心念微细会造成很高的定。更深的定,气就要升得更高。所以,观心念生灭的时候,会发现头上痒痒发胀,有人形容头上像喷泉一样喷水,有些人说好象头上在冒烟。有些人眼睛能看到修观的人头上升起一束光。当修行人的心念不稳定的时候,他头上的亮光会摆来摆去,骗不了人的。当他心念很定的时候,那个光柱会冲上天去。所以真修行者,修行时会发出光,能看见光的人就会看到。所以,真修行时,会来引鬼神,因为鬼神也见到修行人的光。定和慧的气正好颠倒,禅定的气因为集中凝聚,就会留在头上。修观的气是冒上去,结果呢?修定很强的人,他没有办法修观,观太强的人没有办法修定,这两股气是相反的。定力太强的人就没办法修观。为什么?因为那股定的气带他入定,定力太强,观不了,他只好入定了,别人还以为他修得好呢!

\subsection{定力太强障观心}

如果不先去分清楚,你究竟是修观还是修定。你就在那里修啊修啊,不知道修什么。要弄清楚啊,修定会收集能量,修观会消耗能量,越观他会越觉得没有体力,有些人会有这种现象。所以,你很精进修观的时候,会累、会疲倦,疲倦了你精进的心还在,于是身心互相矛盾。其实,修观时要多休息,修定的人应该少睡。大多数人不是观太强,就是定太强。很多人不知道修观时有一个很重要的道理,就是修观时千万不要入定太久,那样会障碍修观。为什么?因为你观没多久就入定了。这样的定力不利于修观。不明白的人还以为这个人很用功在修行,外表看这个人修得很好,在明眼人看来,他一点都没有进步。为什么?他想修观,可是观一点也没修就入定去了。结果是定力越来越强,最后只会发神通,不能开智慧,知道吗?别人看他修出神通而非常崇拜他,在我看来他没进步。不明白修慧的关键在哪里,修来修去,慢慢地越有神通,智慧却没有进展。禅宗祖师最害怕学人入定,因为入定不修慧,是浪费时间。所以,修观的时候一定要注意:不要入定太久。

\subsection{维持四禅定力来观心}

但是,没有足够的定力修不好观,怎么办?你每天一定要能上四禅,早晚各一次。每次要一分钟内就能上四禅,这样就有足够的定力修观。你们的禅定训练好之后,都会有这个经验,两三个呼吸就进去了,有时吸一口气就进四禅了。但是,千万不要在修观时一口气进四禅,那表示你定力太强了,这时修观会一念不小心就进了四禅。定慧都很强的人,可在四禅里修观,一般人往往在定中就不想修观了,弄明白了定慧的心的差别,修观就修得快。不明白的人修观变成入定,在浪费时间,别人还以为他很精进呢!所以你必须每天检查一次,有没有能力立刻再上四禅。如果能够,就证明你今天还有足够定力修观。如果今天五分钟内都上不了四禅,那就暂停修观。把定力再修回来,定力足够了,再来修观,这样用功就用对了,否则,你就是在拖延时间罢了。

记住,修观时,每天早晚一定要检验一次四禅的定力。

\subsection{舍觉支——无心道人}

修观修到身体都没有知觉时,会有一些现象出现。我在这里先不说,说了之后,会障碍你的修行。只有你自己修到那时,才会发现真的有那回事。当你们的心念越来越微细的时候,你头上的气会越冒越高,你会觉得冲上天空去了。当心非常微细时,就是气高到一定时候,你会进入一种无念的状态,在无念时,身心会发生一些古怪的事情,这个就不说出来了。希望已经知道的同学,不要告诉不知道的人,以免障碍他修行,好多人未修到位就先想知道这个过程,结果以后打妄想、自我暗示,以为修成了,自作障啊!

当你达到无念的状态时,你会发现,睁开眼睛看东西,或者做什么事情,你的头都是空空的。其实,每个人都经验过,内心清醒而脑袋一片空白,这时一点儿杂念都没有。我们的心念是这样的:你每天醒来的时候,心的能量就会跑到头上来。睡觉时,头上的能量就会统统退掉。当一个人观到无念的时候,他的头上就会空空的一点儿念头都没有。此时他只要动一个念头,就会发现有一股能量冲上头。就是说,当你修到心念很细,细到不动念的时候,真正的无念必定是头上一点能量都没有,当时动一个念头,想一件事情,就会发现一股能量跑上头来。发生这种现象时,就表示你做到无心道人的状况了。什么叫无心道人呢?面对任何境界,心里知道没有为那个境界动心念。虽然对境界不动心念,五遍行(觉知的心)还是在的,但是,你不思考任何事物,见到心只是一面镜子在照万物,你才知道心对境舍念清净,跟四禅一模一样。你在四禅时,只是在定中没有任何念头。修观修到无念的时候,不必入定就能够任何时候不动念头,进入舍念清净的状况。

当一个人进入这样的状况以后,在平时烦恼心一动,就会使他觉得很辛苦,会有一阵子一阵子的心疼。从此以后,他对贪嗔心等特别警觉,因为嗔心一起来,他立刻觉得心受不了,很辛苦。因为纯净的心已经非常敏感。嗔心一来,心就要疼而受不了,他必须立刻舍掉。这个时候,你要去除贪嗔就比较快。

\chapter{第六讲}

\section{证悟与悟后修}

今天继续讲关于修观的问题。修观就是修慧。

\subsection{增上慢}

当一个人完成初果的时候,他应该自证自知完成初果。但是,修行人往往有一个毛病,就是还没修到那里,就自以为修到某某果位。就自我肯定地说:我已经修到如此这般了,佛法把未证谓证的烦恼称为增上慢。论中说,初果向人有时会有初果的增上慢,二果向人有时会有二果的增上慢,三果向人有时会有三果的增上慢,四果向人有时会有四果的增上慢。增上慢本身不是妄语,他本人不知道,自以为修到那里了。当增上慢人知道自己是增上慢时,他必须承认自己当时自以为是,不然就是大妄语,你有增上慢吗?别掉进去!更要小心大妄语哦!

\subsection{果定}

\subsubsection{现法乐住}

如何验证是否证初果?如果你以为你完成初果了,是否真的就完成了呢?初果是放下我见。在《清净道论》的最后一品,讲到修慧的功德说,如果一个人完成了初果,他就得到一种定,叫果定。果定不是入定,在经典里称果定为现法乐住。一个人如果完成了初果之后,他要享受证果的快乐,随时随地都能够享受,这就是证果的现法乐住,现法乐住比四禅还要舒服。如果你完成了四禅与初果,你就会知道果定的现法乐住比四禅还要乐,还要舒服。这个乐不是世间的快乐,也不是三禅那样的快乐,是一种很平静很舒服的感受。这种感受使你会觉得全身很轻安,一种出离世间的宁静。当时世间的一切你都没有兴趣了,心就自然静静地定在里面。

如果你们真的完成了初果之后,必须是你自信地知道有这么一回事,若要他人印证就是你证初果没信心了。所以,一旦你认为你完成了初果,你应该去作现法乐住,如果你体验到了现法乐住,你就知道及自信有法喜充满这回事,你也会很自信证果了,确信证果的法乐是现法可以乐住享受的。一个人不论是否通过修行完成了初果,也可以尝试做现法乐住,如果他真的完成了初果,他一定有资格(能力)享受证果的快乐,就是说,一定要出现现法乐住。如果他没完成那个果位,就没有办法感受了。现法乐住会随着你的禅定退而退。意思就是说,如果你的禅定退,你的现法乐住也退。

如果你们没有完成四禅,你就不清楚现法乐住的乐是多乐,但是,如果你完成了四禅就会知道,由智慧得到的快乐比修定得来的快乐还要舒服。在经典里面说四禅也有现法乐住。是怎么回事?就是说如果一个人从四禅里出来,他会非常舒服,心念很清净,遇到任何境界,他都会心如止水。如果四禅的定力没了,这个感受也就没了,要重新修回四禅才会有。只要你完成四禅,你就随时享受从四禅出定给你带来的快乐。

在享受现法乐住时,有时会弄错去入定,要分别清楚现法乐住不是入定,不可弄错了,不要集中心念而入定。当有人在做现法乐住的时候,如果你能够看到的话,可以看到他头上会有光向上空冒,那光会升得很高。不需要入定,只要心想享受那快乐,他的光就出现了。这个是骗不了人的。他会法喜充满地在里面享受法乐。

\subsubsection{退现法乐住}

经论里面说有六种阿罗汉,即退法阿罗汉、思法阿罗汉、护法阿罗汉、安住法阿罗汉,堪达法阿罗汉、不动法阿罗汉。其中的退法阿罗汉,就是因为禅定退而退了现法乐住。

可见证了果也会退,很多人不信证果后会退,不信者请参考所例举的经论,不同部派对于退果的见解不同。佛在四十二章经中说:``得阿罗汉道,乃可信汝意耳。'',我不知道四十二章经出自哪个部派,然而,这个部派认为佛说四果以下还有反复心,可见佛只信阿罗汉的心,换句话说,到了三果,心依然会反反复复,就是还会贪染世间,就有可能退果。当然有退也有进,如果你发觉退了,就赶紧修。

\subsection{证果与断烦恼}

\subsubsection{证果已断已知的烦恼}

证初果后还要修的是修道所应断的法:1、欲界贪;2、嗔恚;3、色界贪;4、无色界贪;5、痴;6、慢;7、掉举。就是不必去修那些破除我见的修法了。小乘部派的`说一切有部'的论典里把烦恼分为欲界烦恼,色界烦恼,无色界烦恼,再把三界烦恼个别细分为八十八使见惑及九十八使见思惑,这种讲法已经被现在的北传佛教的学者接受。然而,在经(不是论)里没有九十八使见思惑的说法,只是把烦恼分为五上分结和五下分结。欲界里的五个烦恼,称为五下分结。欲界以上的五个烦恼叫五上分结,五上分结就是色界的烦恼和无色界的烦恼。

五下分结就是我见、戒禁取见、疑、欲界的贪和嗔。

贪心有欲界贪、色界贪和无色界贪。嗔心唯欲界有。痴心通三界。

五上分结就是色界的贪、无色界的贪、掉举、我慢和痴。其中:

须陀洹果是身见、戒取、疑三结已断已知,欲界贪和嗔未断,还重所以来回欲界七次。

斯陀含果修或不修梵行,只断三结、欲界贪和嗔薄,只来欲界投生一次。

阿那含果必修诸梵行,断五下分结,即欲界贪已断,所以不来欲界。

阿罗汉果必修八正道,断五上分结,不来三界。

初果只断三结,二果的欲界的贪和嗔虽薄,却不是完全没有,所以不可以欲界的贪和嗔来验证初果及二果。唯有断五上分结的三果,才可以依有没有断贪和嗔来验证。因此初果虽断我见,还有贪嗔痴、我执及怕死。

一个人完成初果是断五下分结其中的三个结,并没有断除欲界的贪和嗔。证初果之后,欲界的贪还是很重的,有欲界贪就有嗔心。如果你不明白证初果只是破我见没有断其它烦恼,就会怀疑那些已经证初果的人,为什么还有这样多的烦恼呢?那是因为你不通佛法,以为放下我见就什么烦恼都没有了。证初果不必断除贪嗔的烦恼,只是放下我见,放下我见是很容易的。因为你没有弄清楚,于是以为证初果是很难的。其实,所有的果位里,证初果是最快的,我见是一刹那就能放下的。但是,你要懂得怎么去修。你们在修的时候,我会一直鼓励并教导你们,要完成初果不难,但是,如果你去跟人家讲说:某某法师说证悟初果很容易,人家就会骂我。他们认为这是很难的。其实在佛教界里有很多人完成了初果,他还不敢相信自己能证,这些人总觉得他已经有了一定的证悟,但却不敢承认是初果。因为他没办法分辨,他不知道初果断什么烦恼,其实他是证悟了初果的,但是因为听人家讲初果是如何的不得了,比如传说初果者不会踏死虫蚁了!于是,自己虽然有所觉悟却不以为是证悟初果。这种人,你让他做现法乐住,他会立刻就做出现法乐住来的。

\subsection{禅定里也有掉举}

完成初果以后再继续修,就不再是放下我见了,而是放下五下分结里面的另外两个结——欲界贪和嗔。就是说,一个人完成了初果,他还有掉举等五上分结的烦恼。你们会说,入了禅定不是没有掉举了吗?不是的,入定只是没有掉悔,在所有的禅定里面都可以有掉举。很多人不知道,以为修了禅定就没有掉举。进了初禅没有掉悔,但是还有掉举。掉悔和掉举的区别是,掉悔是一个人心中后悔、良心责备自己做错的一些事情。然后他整天想我是一个罪人,我很罪恶,这样,他就没办法入初禅了。所以,掉悔会障碍修初禅。如果一个人整天很后悔,认为自己很罪恶,整天忏悔,是没有办法进初禅的。掉举是整天回忆某一件事情,回忆某一境界。比如说你修完初禅要上二禅时,专心看呼吸、专心观所修的境界,掉举心会叫你回到初禅,又在享受初禅的快乐。就是说你修了初禅以后,执着初禅而产生掉举,你就整天回忆初禅的境界。所以,掉举很严重的人上不了更高的定,可是他不明白是掉举致使他留在原来的定。他反而会说禅定中怎么会有掉举呢?他不知道掉举也发生在五上分结,是色界和无色界里最严重的烦恼。所以,我们要弄清楚,如果进某一个定后心不愿意出来,要修下一个定却又溜回原来的定,就是你对该禅定有贪——色界贪。这个贪就让你掉举。另外,就是痴。痴就是无明,就是还有很多东西不知道。完成初果以后的人会发现我慢是很严重的烦恼。你修得越深入,越发觉别人修行不如你,这就是我慢的烦恼。心念看得越清楚就越我慢,会越来越重。

除非是阿罗汉,不然就有我慢,不能以我慢来评论前三果。

\subsection{应该多观遍行五}

有些修行者观心念生灭,观到正念现前,后来因为做工谋生等种种原故而失去正念,原因是定力退了。这种人必须先修好四禅,然后再重新观五遍行。不要以为禅定和遍行五已经修过了不必再修。重观五遍行是为了把你的微细心重新建立起来,重新掌握看到第一念,然后才开始观心念生灭。如果你对第一念的遍行心所还没有看清楚就去观心念生灭,那是自己欺骗自己。所以,练习观察遍行心所很重要。如果你观好遍行心所,你就可以去观心念的生灭。

\subsection{到烦恼的环境观心}

一旦你认识到无心的时候,你可以躲起来修一阵子,训练保持它。过后你必须出来面对境界,然后去掌握这个心时时保持在无心状态,一旦烦恼心动就立刻知道,然后把烦恼放下。所以,一般人修行进入不同阶段,需要到不同环境磨炼,开始时在修行道场里,一切安排好好地不令心烦乱,这样的修行是温室里面栽培出来的花朵。修行人达到无心后,必须去面对暴风雨考验,若开始时你受不了暴风雨,可以先在温室里努力修一阵子的,最后还是要去面对令你烦恼的环境,如果你逃避令你烦恼的环境,你依然会面临起烦恼的一天。

\subsection{八正道的生活环境}

证阿罗汉果可能吗?在南传佛教国家有可能,中国寺院不太可能,因为阿罗汉是明天死都无所谓了,没得吃都无所谓了。所以,阿罗汉不为生活操心,但是他要过八正道的生活。当今很难看到符合八正道生活环境的寺院道场,在寺庙里的种种事务根本不适合阿罗汉来做;一般寺庙里有很多不符合戒律的生活,杂务太多而且干扰修行的生活。大多数修解脱道者,需要的是远离繁琐的世俗,要有自己的时间,所以要躲到一个比较自由的地方去修。在现代的寺庙里很难自修,寺庙里只适合一起共修。寺院安排的时间、那种生活、还要办事情,很难与你的修法相应,你不能安排符合你自己的功课。比方说,在早课前入了定,三小时后出定,你要被骂了。嘿,你这懒惰鬼,不来做早课啊,吃饭时间你不来吃啊!当你修观修到很投入的时候,有人来打岔了。所以,我说寺庙生活只适合大家一起共修,可是修行进度是不共的,寺庙共修不适合个别人修行的生活。如果,你要磨练你的心念,你要做无心道人,在寺庙里可以磨练。但是,你要完成阿罗汉果,在寺庙就很难。寺庙里很多事情令你无法完成阿罗汉果。比方说,寺庙里有抽签,阿罗汉不可能做这种事情。还有,寺庙里由出家人自己煮饭来吃,阿罗汉只会应供,不吃僧人煮的食物,凡夫比丘认为他是白吃的懒人。不可以讲太多,讲太多就是讲寺庙的坏话。

就是说寺庙已经形成那种生活方式,所以说你要完成阿罗汉果,寺庙生活是不适合的。但是,在南传佛教的一些国家的寺庙,就适合,在大陆一些持戒的道场也比较适合。也就是说持戒的道场要过堂、行堂,出家人不可以做吃行堂。出家人行堂就是犯戒。所以,很多因素使到阿罗汉要离开寺院。如果你要完成阿罗汉果,你要过八正道生活,你只能独居,只能住山。但住山还是有问题,阿罗汉不会自己煮饭。山上谁供养啊。住在城市,你去乞食就会被公安抓去,这要看你们的缘了。

\subsection{入空定}

如果继续修下去,还有一个阶段要修。那个阶段是很多修行的方法都能达到的。就是你会进入一种光明,这个光明很难分辨,怎么说很难分辨呢?因为有些人在初禅就看到光明了。有些人在初禅到四禅都会看到不同的光明,在观心念生灭的当时也可以看到光明。在论典里说证阿罗汉果前要入金刚喻定,在南传佛教里面叫做证入空定,就是要证空性。入空定的时候,会见到一种光明。他是怎么进入这个光明的呢?不允许从禅定入,四禅八定无法进金刚喻定,必须用修观来进入,要以智慧观心念来进入。进这个定之前有一个现象,如果你们有兴趣的话,你们可以去查一些人死了又回来的死亡记录,看看他们怎么经过死亡的经历,证入空定类似死亡的经历。证入空定者先有脱离世间的现象,最后会出现光明,在南传佛教里有人说这是证了空定。你们可以去查看一些早期禅宗祖师讲过的一些人怎么样进入涅盘。也会讲到这一点。换句话说,涅盘就是要懂得怎么去死亡,就是必定先以智慧觉悟不生不死,然后入涅盘时先感觉身体的死亡,最后证实心原来是不生不死,明明了了脱离了身心。也就是密宗里面讲到的,在死亡的阶段会经过法性光明的阶段,而且说在当时若心不迷以为死亡,则可以证悟空性,但死亡时不迷,能做得到的人太少了。如果你们继续修下去的话,你们可以去体验证入空定。

\subsection{问答}

问:进入空定出来的人会怎么样?

师答:进入那种状况的人出来以后,他会有一种感觉,世上没有什么好修的。但是如果按照经典的说法,他出来以后觉得无修无证,就是所作已办。他更应该知道一件事情,就是知道没有下一生。如果他以为无修无证就是所作已办,还不能确定自己没有下一生,其实他还未完全解脱。

问:一个人证了阿罗汉果,他还会不会退?

师答:不同的部派有不同的说法。你们有兴趣的话,可以去看《异部宗轮论》,里面记载了各派对于证果的不同的见解。这些不同的见解在讨论证果后会不会退呀,怎么样才叫证果呀,大家有不同的说法。有的部派说初果不会退,理由是初果属于慧,不属于修道断的烦恼。认为见道时是智慧断见惑,一见之后就永远见,不会退。但是有些部派说初果也会退。至于阿罗汉会退的说法也有几种,南传佛教说不退,其它有不少部派说阿罗汉会退,而且退的等次不同。(请参考后面所录的《中阿含·大品福田经》)有些部派则说阿罗汉会退到初果为止,因为他们说初果绝对没得退。但有些部派说阿罗汉会退到连初果也没有了。另外,有些部派还说,阿罗汉只是退了现法乐住,但是死的时候还是阿罗汉;因为病、因为种种因缘,定力退了,一些烦恼又生起来而退现法乐住,但是他们说阿罗汉绝对不要那些烦恼了,只是因为他的定力不够,那烦恼浮现而已。各种说法都有。

问:初果还会轮回吗?

师答:还要呀。最多七次呀。

问:他还会不会去做畜生呀?

师答:不会了。各部派都说初果绝对不会堕落了。

问:他还会遇到佛法吗?

师答:根据初果的因缘,他一定会继续修行,一定会遇到佛法,不然他无法在七次轮回之内解脱。如果你害怕受轮回苦,证了初果后可以发愿往生极乐世界。往生什么品知道吗?至少中品啊。下品是凡夫,中品是证果的人,上品是大菩萨。

问:那罗汉呢?

师答:罗汉如果发心成为大阿罗汉,这种罗汉还是要再来这个世间。但是有几种说法。一种说法是,大阿罗汉不是以应化身再来这个世间,是以神通变化身来。但是中国人相信,阿罗汉还会继续以应化身来。比如说济公活佛就是个例子,还再来投胎,又来渡众生,继续与众生结缘,行他的菩萨道。

问:他还会迷吗?

师答:多数还会迷。虽然有隔阴之迷,他还会继续去修行,而且很快又觉悟。不但罗汉如此,完成初果者甚至转世到没有佛法的地方,自己也会修行的。因为业习的力量的推动,他会去修行。先碰到外道,就先修外道法,比如练气功或修道教。

问:如果他已经证到了一种果位,他再来的时候,一定还会证到这个果位吧?他还会遇到佛法吗?

师说:能证果就说明与佛法因缘很深,一定会遇到佛法的。

问:入定和中阴身看到的光一样吗?

师说:在入定时也能看到光,中阴身也可以看到光。但是不一样的。大家看见的都是光,没有办法分辨。所以,你要分辨清楚的话,你就必须把四禅八定都修完。把所有的不同的定中看到的光都弄清楚,你才能分别清楚。而且相同的定在不同时候会看到不同的光明。

问:如何修灭尽定?

师说:修观达到无心过后,可以进一步修五、六、七、八的无色定。如果你定力很好,修完五、六、七、八定,最后你可以进入灭尽定。能进入灭尽定的人肯定至少是三果。进入灭尽定时,有人会头低下来。其实,观心生灭观到无心时,有时头也会低下来。原因是高度无心时,心就不控制身体,颈项失去控制而头就会低下来。所以,有些入无心定的道人,当他的头低下来的时候,别人还以为他在打瞌睡。懂得的人一靠近就知道了,一个人进入灭尽定或无想定时,他的周围会产生很强的磁场,靠近此人就会有想入定的感觉。四禅八定都是有心定,入定、出定、在定都自心明了。灭尽定是无心定,无心定是入定、出定、在定都无心。灭尽定的步骤:一开始上坐的时候,先心想我现在要进入灭尽定,然后入四禅八定里观生灭,之后你再不可以有进入灭尽定这个念头。因为是无心定,不可以象四禅八定那样心想进就进。只有三果以上的某些定解脱的圣人,到四禅八定以后自然地进入灭尽定。另外,灭尽定是无心定,他出来的时候也是无心出定。

\appendix

\chapter{禅定相关经典}

在修法上一定要依老师的指示去修,这时依人不依法。证悟时,必须是自证自知,请勿依靠老师的印证,也不可以找非同见同行的人释疑,这时,应当依据经典,依法不依人。以下是应读的经典:

注意经文说:初禅的人听到声音语言便退失,四禅呼吸断。

\subsection{长阿含第二分众集经第五}

比丘除欲恶不善法。有觉有观,离生喜乐,入于初禅。灭有觉观,内信一心,无觉无观,定生喜乐,入第二禅。离喜修舍,念进,自知身乐,诸圣所求,忆念、舍、乐,入第三禅。离苦乐行,先灭忧喜,不苦不乐,舍念清净,入第四禅。

\subsection{长阿含经卷第十二 第二分清净经第十三}

有觉、有观,离生喜乐入初禅。如是乐者佛所称誉。犹如有人灭于觉观,内喜一心,无觉无观,定生喜乐入第二禅。如是乐者佛所称誉。犹如有人除喜入舍,自知身乐,贤圣所求,护念一心入第三禅。如是乐者佛所称誉。乐尽苦尽,忧喜先灭,不苦不乐,护念清净入第四禅。如是乐者佛所称誉。

\subsection{杂阿含(747)禅定灭何法}

如是我闻

一时。佛住王舍城迦兰陀竹园。

尔时。尊者阿难独一静处禅思。念言``世尊说三受,乐受、苦受、不苦不乐受,又复说诸所有受悉皆是苦。此有何义?''作是念已,从禅起诣世尊所,稽首礼足,退住一面,白佛言:``世尊,我独一静处禅思,念言`如世尊说三受,乐受、苦受、不苦不乐受,又说一切诸受悉皆是苦。此有何义?'\,''

佛告阿难:``我以一切行无常故,一切行变易法故,说诸所有受悉皆是苦。又复,阿难!我以诸行渐次寂灭故说,以诸行渐次止息故说,一切诸受悉皆是苦。''

阿难白佛言:``云何?世尊,以诸受渐次寂灭故说?''

佛告阿难:``初禅正受时,言语寂灭。第二禅正受时,觉观寂灭。第三禅正受时,喜心寂灭。第四禅正受时,出入息寂灭。空入处正受时,色想寂灭。识入处正受时,空入处想寂灭。无所有入处正受时,识入处想寂灭。非想非非想入处正受时,无所有入处想寂灭。想受灭正受时,想受寂灭。是名渐次诸行寂灭。''

阿难白佛言:``世尊,云何渐次诸行止息?''

佛告阿难:``初禅正受时言语止息,二禅正受时觉观止息,第三禅正受时喜心止息,四禅正受时出入息止息。空入处正受时,色想止息。识入处正受时,空入处想止息。无所有入处正受时,识入处想止息。非想非非想入处正受时,无所有入处想止息。想受灭正受时,想受止息。是名渐次诸行止息。''

阿难白佛:``世尊,是名渐次诸行止息。''

佛告阿难:``复有胜止息、奇特止息、上止息?无上止息。如是止息,于余止息无过上者。''

阿难白佛:``何等为胜止息、奇特止息、上止息、无上止息。诸余止息无过上者?''

佛告阿难:``于贪欲,心不乐,解脱。恚、痴,心不乐,解脱。是名胜止息、奇特止息、上止息、无上止息。诸余止息无过上者。?

佛说此经已。尊者阿难闻佛所说。欢喜奉行。

\subsection{中阿含经(204)晡利多品罗摩经}

世尊答曰。优陀夷。比丘离欲.离恶不善之法。有觉.有观。离生喜.乐。得初禅成就游。得共彼天戒等.心等.见等也。彼觉.观已息。内靖.一心。无觉.无观。定生喜.乐。得第二禅成就游。得共彼天戒等.心等.见等也。彼离于喜欲。舍无求游。正念正智而身觉乐。谓圣所说.圣所舍.念.乐住.室。得第三禅成就游。得共彼天戒等.心等.见等也。优陀夷。是谓一道迹一向作世证

\subsection{经曰:初禅以声为刺,四禅以入息出息为刺}

中阿含长寿王品无刺经第十三(第二小土城诵)

入初禅者以声为刺。入第二禅者以觉观为刺。入第三禅者以喜为刺。入第四禅者以入息出息为刺。入空处者以色想为刺。入识处者以空处想为刺。入无所有处者以识处想为刺。入无想处者以无所有处想为刺。入想知灭定者以想知为刺。

复次。有三刺:欲刺.恚刺.愚痴之刺。此三刺者。漏尽阿罗诃已断.已知。拔绝根本。灭不复生。

\subsection{十诵律卷第二}

律曰:目揵连自说入无色定见色闻声

一时长老大目揵连。在耆阇崛山入无所有空定。善取入定相。不善取出定相。从三昧起。闻阿修罗城中伎乐音声已。还疾入定。作如是念。我在定中闻阿修罗城中伎乐音声。从三昧起语诸比丘。我在耆阇崛山入无所有处无色定。闻阿修罗城中伎乐音声。诸比丘语目连。何有是处。入无色定当见色闻声。何以故。若入无色定。破坏色相舍离声相。汝空无过人法故作妄语。汝目连应摈治驱遣。是事白佛。佛语诸比丘。汝等莫说目连犯罪。何以故。目连但见前事不见后事。如来亦见前亦见后。是目连在耆阇崛山,入无所有处无色定。善取入定相,不善取出定相,从定起闻阿修罗城中伎乐音声,闻已还疾入定。便自谓,我入定闻声。若入无色定。若见色若闻声无有是处。何以故。是人破坏色相舍离声相故。若目连空无过人法故妄语者亦无是处。是目连随心想说无罪。

\chapter{证果相关经典}

\subsection{关于证果}

证果到底要断什么烦恼?在经典只说到十结,即五上分结和五下分结。佛教发展到部派分裂而对证果众说纷纷,结果有部论师创立88使和98使的复杂理论,如今中国佛教几乎全盘接受88使和98使的讲法,更创立81惑。然而,佛在经典里不曾说到有88使和98使。而且南传佛教如今依然只是讲十结,我个人认为中国佛教应当放弃88使和98使的理论,以五上分结和五下分结来讲解证果。

以下列出88使和98使的理论出自哪些论典,编号引用大正藏的号码,其中1521及1527是大乘论师依据小乘论典而说98使,1541/1543/1546/1547/1552都是小乘论师的著作,不是佛所说的论典。

1521 十住毘婆沙論  聖者龍樹造 98

1527 涅槃論 婆藪槃豆作 98

1541 眾事分阿毘曇論 尊者世友造 88,98

1543 阿毘曇八犍度論 迦旃延子造 88,98

1546 阿毘曇毘婆沙論 迦旃延子造 五百羅漢釋 88,98

1547 鞞婆沙論 阿羅漢尸陀槃尼撰 88,98

1552 雜阿毘曇心論  尊者法救造 88,98

\subsection{律曰:须陀洹不复受三涂报,于无上道决定信}

善见律毘婆沙卷第十八

如是我闻。一时佛在迦毘罗卫国尼拘陀林。时释摩男。与五百优婆塞。往诣佛所。顶礼佛足。在一面坐。白佛言。世尊。如佛所说。优婆塞义。在家白衣。具丈夫志。归命三宝。自言我是优婆塞者。云何而得须陀洹果。乃至阿那含耶。佛告释摩男。断除三结。身见戒取及疑网等。断三结已。成须陀洹。更不复受三涂之身。于无上道。生决定信。人天七返。尽诸苦际入于涅盘。是名优婆塞得须陀洹。又问。云何而得斯陀含果。佛告摩诃男。断三结已,薄淫怒痴,名斯陀含。又问。云何而得阿那含果。佛告摩诃男。若能断三结,及五下分,成阿那含。时摩诃男及五百优婆塞。闻此法已。心生欢喜。而白佛言。世尊。甚为希有。诸在家者。获此胜利。一切咸应作优婆塞。时摩诃男及诸优婆塞。作是语已。礼佛而退。诸比丘等闻佛所说。欢喜奉行。

\subsection{律曰:证果与所断烦恼。没说88使98使}

毘尼母经卷第八

见谛中所应断者。有六。一身见。二疑。三戒取。四向恶道欲。五向恶道恚。六向恶道痴。修道所应断。一欲染。二恚。三色染。四无色染。五无明。六慢。七调。断如此七烦恼。便得证果。断三结得须陀洹。欲染恚薄故得斯陀含。欲染恚断故得阿那含。一切结尽故名阿罗汉。

\subsection{经曰:证果与所断烦恼。没说88使98使}

佛说长阿含第二分自欢喜经第十四

如来说法复有上者。所谓教诫。教诫者,或时有人不违教诫,尽有漏成无漏,心解脱、智慧解脱,于现法中自身作证,生死已尽、梵行已立、所作已办、不复受有,是为初教诫。或时有人不违教诫,尽五下结,于彼灭度不还此世,是为二教诫。或时有人不违教诫。三结尽,薄淫、怒、痴,得斯陀含,还至此世而取灭度,是为三教诫。或时有人不违教诫,三结尽,得须陀洹,极七往返,必成道果,不堕恶趣。是为四教诫。此法无上。智慧无余。神通无余。诸世间沙门、婆罗门无有与如来等者。况欲出其上

\subsubsection{放光般若经 摩诃般若波罗蜜学五眼品第四}

以解脱慧度于三碍:有身碍、有狐疑碍、有邪信碍,度是三碍得须陀洹。便道得念,于淫、怒、痴薄,得斯陀含。精勤于道,却淫、怒、痴,得阿那含。便消五爱:一者色爱、二者无色爱、三者痴爱、四者恨戾爱、五者乱志爱,已度是者便得罗汉。如是行空菩萨,便得空脱,便成五根。疾近不中止禅,至罗汉道。是人已得无相解脱,逮得五力乃至罗汉。是为菩萨得法眼净。

\subsubsection{佛为年少比丘说正事经}

有比丘三结尽得须陀洹,不堕恶趣,法决定向三菩提,七有天人往生究竟苦边。有比丘三结尽,贪、恚、痴薄,得斯陀含。有比丘五下分结尽,得阿那含,生般涅盘,不复还生。此世有比丘得无量神通境界,天耳、他心智、宿命智、生死智、漏尽智。

\subsubsection{杂阿含经(393)}

如是我闻。一时。佛住波罗奈国,仙人住处鹿野苑中。尔时。世尊告诸比丘。若善男子,正信非家,出家学道,彼一切所应,当知四圣谛法。何等为四:谓知苦圣谛、知苦集圣谛、知苦灭圣谛、知苦灭道迹圣谛。是故比丘,于四圣谛未无间等者,当勤方便修无间等。如此章句,一切四圣谛经,应当具说。佛说此经已,诸比丘闻佛所说,欢喜奉行。如是知、如是见、如是无间等,悉应当说。

又三结尽,得须陀洹。一切当知四圣谛,何等为四:谓知苦圣谛、知苦集圣谛、知苦灭圣谛、知苦灭道迹圣谛。如是当知、如是当见,无间等。

若三结尽,贪、恚、痴薄,得斯陀含。彼一切皆于四圣谛如实知故。何等为四。谓知苦圣谛、知苦集圣谛、知苦灭圣谛、知苦灭道迹圣谛。如是当知、如是当见,如是无间等。亦如是说。

五下分结尽,生般涅盘阿那含。不还此世。彼一切知四圣谛。何等为四。知苦圣谛、知苦集圣谛、知苦灭圣谛、知苦灭道迹圣谛。如是知、如是见,如是无间等。亦如是说

若一切漏尽,无漏心解脱、慧解脱,见法自知作证,我生已尽、梵行已立、所作已作、自知不受后有。彼一切悉知四圣谛,何等为四:谓知苦圣谛、知苦集圣谛、知苦灭圣谛、知苦灭道迹圣谛。如是知、如是见,如是无间等。亦如是说。

\subsubsection{杂阿含经(797)}

如是我闻。一时。佛住舍卫国祇树给孤独园。尔时。世尊告诸比丘。有沙门法及沙门果。谛听。善思。当为汝说。何等为沙门法?谓八圣道,正见乃至正定。何等为沙门果?谓须陀洹果、斯陀含果、阿那含果、阿罗汉果。何等为须陀洹果?谓三结断。何等为斯陀含果。谓三结断。贪、恚、痴薄。何等为阿那含果。谓五下分结尽。何等为阿罗汉果,谓贪、恚、痴永尽,一切烦恼永尽。

\subsubsection{杂阿含经(928)}

摩诃男白佛言。世尊。云何名优婆塞须陀洹

佛告摩诃男。优婆塞须陀洹者,三结已断已知,谓身见、戒取、疑。摩诃男。是名优婆塞须陀洹

摩诃男白佛言。世尊。云何名优婆塞斯陀含

佛告摩诃男。谓优婆塞三结已断已知。贪.恚.痴薄。摩诃男。是名优婆塞斯陀含

摩诃男白佛言。世尊。云何名优婆塞阿那含

佛告摩诃男。优婆塞阿那含者。五下分结已断已知。谓身见.戒取.疑.贪欲.瞋恚。摩诃男。是名优婆塞阿那含

\subsection{经曰:阿罗汉有退,9种阿罗汉}

中阿含大品福田经第十一

我闻如是。一时,佛游舍卫国,在胜林给孤独园。尔时,给孤独居士往诣佛所,稽首佛足,却坐一面。白曰:世尊!世中为有几福田人?

世尊告曰:居士!世中凡有二种福田人,云何为二?一者学人,二者无学人。学人有十八,无学人有九。

居士!云何十八学人?信行、法行、信解脱、见到、身证、家家、一种、向须陀洹、得须陀洹、向斯陀含、得斯陀含、向阿那含、得阿那含、中般涅盘、生般涅盘、行般涅盘、无行般涅盘、上流色究竟。是谓十八学人。

居士!云何九无学人?思法、升进法、不动法、退法、不退法、护法护则不退。不护则退、实住法、慧解脱、俱解脱。是谓九无学人

\subsection{律曰:阿罗汉有退,9种阿罗汉}

根本说一切有部尼陀那卷第二

长者以诸衣物置上座前。即便前礼佛足白言。世尊。于此人间几是福田。佛言。有二。谓学及无学。学人差别有十八种。无学之人有其九种。是谓福田。堪销物利。云何十八种有学人?谓预流向、预流果、一来向、一来果、不还向、不还果、阿罗汉向、随信行、随法行、信解、见至、家家、一间、中、生、有行、无行、上流。是名十八。何等名为九种无学?谓退法、思法、护法、住法、堪达法、不动法、不退法、慧解脱、俱解脱。是名为九。

\subsection{律曰:四三二果有退。得果退者言退失不犯}

萨婆多部毘尼摩得勒伽卷第八

问:若比丘如是语。我于四沙门果退。犯何罪。

答:偷罗遮。

问:若比丘言。我得四沙门果。得何罪。

答:不得言得。波罗夷罪

问:我失阿罗汉果。阿那含果。斯陀含果。得何罪耶。

答:不犯。实不得退言得退。波罗夷。实者不犯

问:若比丘言我是学人。得何罪耶。

答:若言我学波罗提木叉。偷罗遮。若空无所有言学圣法。波罗夷。学修多罗毘尼阿毘昙亦如是。

若比丘言。我是最后生。犯何罪耶。

答:若说过去法已灭。偷罗遮。若说实生尽。波罗夷

\subsection{律曰:二果失精,不说三果失精}

四分律卷第二

世尊以此因缘即集诸比丘告言。乱意睡眠有五过失。一者恶梦。二者诸天不护。三者心不入法。四者不思惟明相。五者于梦中失精。是为五过失。善意睡眠有五功德。不见恶梦。诸天卫护心入于法。系意在明相。不于梦中失精。是谓五功德。于梦中失精不犯。精有七种。青黄赤白黑酪色酪浆色。何者精青色转轮圣王精也。何者精黄色转轮圣王太子精也。何者精赤色犯女色多也。何者精白色负重人精也。何者精黑色转轮圣王第一大臣精也。何者精酪色须陀洹精也。何者精酪浆色斯陀含人精也。

\subsection{律曰:讲解入禅得果无犯}

四分律卷第十七~~ 说云何得证果

尔时佛在舍卫国祇树给孤独园。尔时十七群比丘往语六群比丘:``长老,云何入初禅、第二、第三、第四禅?云何入空无相无愿?云何得须陀洹果、斯陀含果、阿那含果、阿罗汉果耶?''时六群比丘报言:``如汝等所说者,则已犯波罗夷法,非比丘。''时十七群比丘,便往上座比丘所问言:``若有诸比丘作如是问`云何入初禅、二禅乃至四禅,空、无相、愿,须陀洹乃至阿罗汉果,为犯何罪?'\,''上座比丘报言:``无所犯。''十七群比丘言:``我等向者诣六群比丘所问言`云何入初禅乃至四禅,空、无相、愿,云何得须陀洹果乃至阿罗汉果'彼即报言`汝等自称得上人法,犯波罗夷非比丘。'\,''彼比丘即察知,此六群比丘与十七群比丘作疑恼。

\subsection{律曰:证果向他人说,犯波夜提罪}

十诵律卷第十~~~ 不得实说过人法

是中犯者:

若比丘,实是阿罗汉,向他人说,波夜提。

实向阿罗汉,向他人说,波夜提。

实阿那含、向阿那含,实斯陀含、向斯陀含,实须陀洹、向须陀洹,向他人说,皆波夜提。

若比丘,实得初禅,向人说者,波夜提。

实得二禅、三禅、四禅,慈、悲、喜、舍,空处、识处、无所有处、非有想非无想处,不净观、阿那般那念,向他人说,波夜提。

乃至我好持戒,向他人说,突吉罗。

若比丘实见诸天来至我所。龙、夜叉、浮荼鬼、毘舍遮鬼、罗剎鬼来至我所,向他人说,波夜提。

乃至实见土鬼来至我所,向他人说,突吉罗

\subsection{律曰:证果不一定会说法}

十诵律卷第十一

佛在舍卫国,尔时,佛告诸比丘:我教化四众疲极,令诸比丘当教诫比丘尼。尔时,诸比丘受佛教已,次第教诫比丘尼。上座比丘次第教诫竟,次至长老般特。

时阿难往语般特言:``汝知不,汝明日次应教诫比丘尼。''般特语阿难言:``我钝根不多闻,未有所知,我夏四月乃能诵得一拘摩罗偈`智者身口意不作一切恶。常系念现前舍离于诸欲。亦不受世间无益之苦行'阿难,得过是次者善。''

阿难再三语般特言:``诸上座已教诫竟,今次到汝。''般特比丘亦再三报阿难言:``我钝根不多闻,未有所知,夏四月乃能诵得拘摩罗一偈。得过次者善。''

阿难复言:``汝明日次教诫比丘尼。''即受阿难语。夜过已,中前着衣持钵,入舍卫城次第乞食,食后还自房舍,空地敷坐床已入室坐禅。

尔时诸比丘尼闻,今日般特比丘次教诫比丘尼,皆生轻心:``是不多闻诵读经少,夏四月过诵得一拘摩罗偈`智者身口意不作一切恶,常系念现前舍离于诸欲,亦不受世间无益之苦行'。我等所未闻法云何得闻?我等所未知法云何得知?所诵拘摩罗偈,我等先已诵。''

诸有比丘尼,先不入祇陀林听法者,时皆共来。有五百比丘尼,出王园比丘尼精舍,往祇桓听法。诣长老般特房前立,謦欬作声扣户言:``大德般特,出来!''长老般特即从禅起出房,至独坐床上,端身大坐,诸比丘尼头面礼竟,皆在前坐。

时长老般特以柔软语言:``诸姊妹!当知我钝根少所读诵,夏四月过诵得一偈`智者身口意,不作一切恶,常系念现前舍离于诸欲,亦不受世间无益之苦行'虽然我当随所知说,汝等当一心行不放逸法。何以故?乃至诸佛,皆从一心不放逸行,得阿耨多罗三藐三菩提。所有助道善法,皆以不放逸为本。''

作是语已,用神通力于座上没,在于东方虚空之中,现四威仪行立坐卧。入火光三昧身出光焰,青黄赤白种种色光。身下出火身上出水,身下出水身上出火,南西北方四维上下亦复如是。种种现神力已还坐本处。

诸比丘尼见长老般特如是神力已,轻心灭尽,生信敬心故尊重净心,折伏憍慢。即随比丘尼所憙乐法所应解法,而为演说。众中有得须陀洹果、斯陀含果、阿那含果、阿罗汉果。有种声闻道因缘,有种辟支佛道因缘,有发阿耨多罗三藐三菩提因者。尔时众中。得如是种种大利益。

\subsection{经曰:一聚落内有500阿那含。}

由本经这说明有无数人于佛前证初果,这些人于末法都会天上人间来回地继续修行证果。别听人瞎说此时不可能修证。

杂阿含经(854)

~~~~世尊。彼等命终。当生何处

佛告诸比丘。彼罽迦舍等,已断五下分结得阿那含。于天上般涅盘。不复还生此世

诸比丘白佛。世尊。复有过二百五十优婆塞命终。复有五百优婆塞于此那梨迦聚落命终。皆五下分结尽。得阿那含。于彼天上般涅盘。不复还生此世。复有过二百五十优婆塞命终。皆三结尽,贪、恚、痴薄,得斯陀含。当受一生。究竟苦边。此那梨迦聚落复有五百优婆塞于此那梨迦聚落命终。三结尽,得须陀洹。不堕恶趣法。决定正向三菩提。七有天人往生。究竟苦边

佛告诸比丘。汝等随彼命终、彼命终而问者。徒劳耳。非是如来所乐答者。夫生者有死。何足为奇。如来出世及不出世。法性常住。彼如来自知成等正觉。显现演说。分别开示。所谓是事有故是事有。是事起故是事起。缘无明有行。乃至缘生有老、病、死、忧、悲、恼苦。如是苦阴集。无明灭则行灭。乃至生灭则老、病、死、忧、悲、恼苦灭。如是苦阴灭。今当为汝说法镜经。谛听。善思。当为汝说。何等为法镜经。谓圣弟子于佛不坏净。于法、僧不坏净。圣戒成就

佛说此经已。诸比丘闻佛所说。欢喜奉行

\subsection{经曰:初果优婆夷畜养男女服习五欲(妓女)}

杂阿含经(964)

~~~~婆蹉白佛。颇有一比丘于此法、律得尽有漏。无漏心解脱。乃至不受后有耶

佛告婆蹉。不但若一。若二、若三。乃至五百。有众多比丘于此法、律尽诸有漏。乃至不受后有

婆蹉白佛。且置比丘。有一比丘尼于此法、律尽诸有漏。乃至不受后有不

佛告婆蹉。不但一、二、三比丘尼。乃至五百。有众多比丘尼于此法、律尽诸有漏。乃至不受后有

婆蹉白佛。置比丘尼。有一优婆塞修诸梵行。于此法、律度狐疑不

佛告婆蹉。不但一、二、三。乃至五百优婆塞。乃有众多优婆塞修诸梵行。于此法、律断五下分结。得成阿那含。不复还生此

婆蹉白佛。复置优婆塞。颇有一优婆夷于此法、律修持梵行。于此法、律度狐疑不

佛告婆蹉。不但一、二、三优婆夷。乃至五百。乃有众多优婆夷于此法、律断五下分结。于彼化生。得阿那含。不复还生此

婆蹉白佛。置比丘、比丘尼、优婆塞、优婆夷修梵行者。颇有优婆塞受五欲。而于此法、律度狐疑不

佛告婆蹉。不但一、二、三。乃至五百。乃有众多优婆塞居家妻子,香华严饰,畜养奴婢。于此法、律断三结。贪、恚、痴薄。得斯陀含。一往一来。究竟苦边

婆蹉白佛。复置优婆塞。颇有一优婆夷受习五欲。于此法、律得度狐疑不

佛告婆蹉。不但一、二、三。乃至五百。乃有众多优婆夷在于居家。畜养男女。服习五欲。华香严饰。于此法、律三结尽。得须陀洹。不堕恶趣法。决定正向三菩提。七有天人往生。究竟苦边。(女人畜养男女,服习五欲,即妓女)

婆蹉白佛言。瞿昙。若沙门瞿昙成等正觉。若比丘、比丘尼、优婆塞、优婆夷修梵行者。及优婆塞、优婆夷服习五欲。不得如是功德者。则不满足。以沙门瞿昙成等正觉。比丘、比丘尼、优婆塞、优婆夷修诸梵行。及优婆塞、优婆夷服习五欲。而成就尔所功德故。则为满足。瞿昙。今当说譬

\subsection{经曰:不梵行能得二果,故不应筹量他人道行}

杂阿含经(990)鹿住优婆夷心生狐疑

如是我闻。一时。佛住舍卫国祇树给孤独园。尔时。尊者阿难晨朝着衣持钵。诣舍卫城。次第乞食。至鹿住优婆夷舍。鹿住优婆夷遥见尊者阿难。疾敷床座。白言。尊者阿难令坐

时。鹿住优婆夷稽首礼阿难足。退住一面。白尊者阿难。云何言世尊知法。我父富兰那先修梵行。离欲清净。不着香花。远诸凡鄙。叔父梨师达多不修梵行。然其知足。二俱命终。而今世尊俱记二人同生一趣。同一受生。同于后世得斯陀含。生兜率天。一来世间。究竟苦边。云何。阿难。修梵行、不修梵行。同生一趣、同一受生、同其后世

阿难答言。姊妹。汝今且停。汝不能知众生世间根之差别。如来悉知众生世间根之优劣。如是说已。从坐起去

时。尊者阿难还精舍。举衣钵。洗足已。往诣佛所。稽首佛足。退坐一面。以鹿住优婆夷所说广白世尊

佛告阿难。彼鹿住优婆夷云何能知众生世间根之优劣。阿难。如来悉知众生世间根之优劣。

阿难。或有一犯戒。彼于心解脱、慧解脱不如实知。彼所起犯戒无余灭、无余没、无余欲尽。或有一犯戒。于心解脱、慧解脱如实知。彼所起犯戒无余灭、无余没、无余欲尽。于彼筹量者言。此亦有如是法。彼亦有是法。此则应俱同生一趣、同一受生、同一后世。彼如是筹量者。得长夜非义饶益苦。

阿难。彼犯戒者。于心解脱、慧解脱不如实知。彼所起犯戒无余灭、无余没、无余欲尽。当知此人是退。非胜进。我说彼人为退分。

阿难。有犯戒。彼于心解脱、慧解脱如实知。彼于所起犯戒无余灭、无余没、无余欲尽。当知是人胜进不退。我说彼人为胜进分。自非如来。此二有间。谁能悉知。是故。阿难。莫筹量人人而取。人善筹量人人而病。人筹量人人。自招其患。唯有如来能知人耳。如二犯戒。二持戒亦如是。彼于心解脱、慧解脱不如实知。彼所起持戒无余灭。若掉动者。彼于心解脱、慧解脱不如实知。彼所起掉无余灭。彼若瞋恨者。彼于心解脱、慧解脱不如实知。彼所起瞋恨无余灭。若苦贪者。彼于心解脱、慧解脱如实知。彼所起苦贪无余灭。秽污清净如上说。乃至如来能知人人

阿难。鹿住优婆夷愚痴少智。而于如来一向说法心生狐疑。云何。阿难。如来所说。岂有二耶

阿难白佛。不也。世尊

佛告阿难。善哉。善哉。如来说法若有二者。无有是处。阿难。若富兰那持戒。梨师达多亦同持戒者。所生之趣。富兰那所不能知。梨师达多为生何趣。云何受生。云何后世。若梨师达多所成就智。富兰那亦成就此智者。梨师达多亦不能知彼富兰那当生何趣。云何受生。后世云何。阿难。彼富兰那持戒胜。梨师达多智慧胜。彼俱命终。我说二人同生一趣。同一受生。后世亦同是斯陀含。生兜率天。一来生此究竟苦边。彼二有间。自非如来。谁能得知。是故。阿难。莫量人人。量人人者。自生损减。唯有如来能知人耳

佛说此经已。尊者阿难闻佛所说。欢喜奉行。

\subsection{佛灭后(公元1500-2500)还能得阿那含}

善见律卷第十八:(公元1500-2500)中得阿那含不难

何以佛不听女人出家?为敬法故。若度女人出家,正法只得五百岁住。由佛制比丘尼八敬,正法还得千年(前500-公元500)。法师曰:千年已,佛法为都尽也?答曰:不都尽。于千年(公元500-1500)中得三达智。复千年(公元1500-2500)中得爱尽罗汉,无三达智。复千年(公元1500-2500)中得阿那含。复千年(公元2500-3500)中得斯陀含。复千年(公元3500-4500)中得须陀洹学法。复得五千岁(公元4500-9500)。于五千岁得道。后五千年学而不得道。万岁后经书文字灭尽。但现剃头有袈裟法服而已。

\subsection{证果的信心、真伪、退果}

修行为了解脱生死,小乘佛教称为证果,共有四果。中国禅宗称为明心见性,后期的祖师更立下三关。佛在世时,已经有人怀疑佛所印证的二果不正确(杂阿含990)。在佛灭后对四果的印证更是议论纷纷,最开始有大天五事中的阿罗汉有无知,后来发展到阿罗汉退果,乃至阿罗汉可退至无果的说法。如今更加难判断修行人所证何果,例如增上慢人的未证以为已证,以及已证果后退果。所以千万别自以为证果而告诉他人,因为你可能是未证以为已证,也可能证了退果,怀疑你的人一旦发现你的烦恼再起时,他们将会耻笑你,甚至于把你赶出三门外。所以要小心不要向人说你已证xx果,也要小心观察证和退,以下是关于证果的问题:

1:真实证果随意向他人说,犯波夜提。

2:观因缘、观五蕴、观生灭、观十八界等,皆可证初果至阿罗汉果。

3:一个修法让他证阿罗汉,不等于你修完成了也是阿罗汉。

4:修行者贪求果位,自以为修证——增上慢。

5:修观后必须经过一番磨炼,才能确定烦恼是否断除。

6:见道可以顿见,所以初果可以速证。入定、念佛、持咒非修慧故难见道,观因缘是破我见最好的修法。

7:修道必须渐修,即贪嗔只能渐断,所以应当经常修观。

8:烦恼不可断灭,以无生缘为断。只要失去正念正知,烦恼就有因缘再度生起。

9:定力退失,则正念也退失。正念退失,则烦恼得生缘。

10:每天早晚检查定力是否退,退了赶快修回来。

11:平时多观因果、因缘、生灭,世间的生灭、无常、苦、无我。

12:唯有过出世间的八正道生活,才能证阿罗汉。

13:世人以人为师,以圣人为依止,真修解脱道者并非如此。佛说:佛灭后,以戒为师,依止四念处。

14:阿罗汉之前三果,心勿与色会,不然祸生。四十二章经曰:佛告沙门。慎无信汝意。意终不可信。慎无与色会。与色会即祸生。得阿罗汉道。乃可信汝意。

15:不要道听途说,对于退果若有疑问,请看《异部宗轮论》或《异部执论》。

16:要以心无烦恼为自在,不要以开悟或证果而自满。

17:要以烦恼减轻、智慧增长为成果。别以你修完一个法门了,以此当成修行成果。

18:众生各有因缘,菩萨行持更是难思议。有人表面上虽未修定慧,可能一入定就是四禅,一修观就证果。所以,别认为你有戒、有定、有慧、有修行,他人无。

19:有缘先开悟或证果,道行不一定比后来者更高。

20:除非你是阿罗汉,不然你还有我慢、贪、嗔、掉举、怕死。

21:你所崇拜的老师绝对不是一切智者,别以为你的老师有修证,就迷信他的一切,以他的话为权威。

22:你想当圣人吗?先请一切人调查与研究你的一切缺点。因为圣人不怕调查,还是先让人证明我无缺点吧!

23:金刚经说证果者不会说:有我得初果,乃至阿罗汉果。

24:不论老少、智愚都能证得四果。大智慧的如舍利弗,最愚笨的如周梨盘陀伽。年龄极老的如一百二十岁的如须跋陀罗,年轻的如七岁沙弥均头。

25:证果的快慢是个人因缘成熟,是极不一致的。阿难从佛身边极久,还没有证罗汉;而舍利弗、憍陈如等,不过几天就成了罗汉。

26:不是证得须陀洹以后,今生努力进修即能得阿罗汉,好多是此生证得初果或二果、三果后,就停顿不前了。

\begin{center}\itshape 全文完\end{center}

\end{document}
